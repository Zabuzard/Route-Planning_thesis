\documentclass[11pt,oneside,bibliography=totoc,abstracton]{scrartcl}

% Resemble a4wide style
\usepackage{geometry}
\geometry{
	includeheadfoot,
 	left=1.5in,
	right=1in,
	top=1in,
	bottom=1in
	%margin=4cm
}

\usepackage[english]{babel} % Document in English language
\usepackage{graphicx}% More options when including graphics
\usepackage[utf8]{inputenc}% Correct font encoding for special characters
\usepackage[T1]{fontenc}% Correct separation for special characters
\usepackage{lmodern}% Modern font type

\usepackage{array}% More table options
\usepackage{verbatim}% New environment, also for comments
\usepackage{listings}% New environment, mainly used for code

\usepackage{fancyhdr}% Can create headers and footers
\usepackage{amssymb}% Provides usage of special characters and symbols
\usepackage{amsthm}% Offers theorem environment
\usepackage{amsmath}% Provides mathematical symbols
\usepackage{mathtools}% Provides some mathematical tools and symbols
\usepackage{semantic}% Provides semantic formats
\usepackage{stmaryrd}% More mathematical symbols
\usepackage[ruled,vlined,linesnumbered,resetcount]{algorithm2e}% Algorithms
\usepackage{color}% Colored text

\usepackage{subcaption}% Makes it possible to create a sub captions in tables
\usepackage{setspace}% Makes it possible to change spacing
\usepackage{textcomp}% Font that provides more special characters
\usepackage{lscape}% Modifies margins
\usepackage{wasysym}% More symbols

\usepackage{tikz}% Offers advanced functions for drawing vector graphics
\usetikzlibrary{patterns}% Provides patterns for filling nodes
\usetikzlibrary{shapes}% Provides shapes for nodes
\usetikzlibrary{calc}% Enables basic math for tikz
\usetikzlibrary{automata,arrows}% Enable automata related commands for tikz
\usetikzlibrary{positioning}% Special commands for positioning paths
\tikzstyle{redArea}=[draw=red!30,line width=1pt,preaction={clip, postaction={pattern=north west lines, pattern color=red!30}}]
\tikzstyle{blueArea}=[draw=blue!30,line width=1pt,preaction={clip, postaction={pattern=dots, pattern color=blue!30}}]
\tikzstyle{greenArea}=[draw=darkgreen!30,line width=1pt,preaction={clip, postaction={pattern=horizontal lines, pattern color=darkgreen!30}}]
\tikzstyle{orangeArea}=[draw=orange!30,line width=1pt,preaction={clip, postaction={pattern=crosshatch, pattern color=orange!30}}]
\tikzstyle{blueBorder}=[draw=blue,line width=1pt,preaction={clip, postaction={draw=blue,opacity=0.5,line width=12pt}}]

\usepackage{hyperref}% Appearance of links do not get changed
\renewcommand{\figurename}{Fig.}% Fig. instead of Figure. when linking figures

\bibliographystyle{plain}% Reference literature by number
\pagestyle{fancy}% Fancy style, also adds header and footer

%Custom colors
\definecolor{purple}{RGB}{206,0,206}
\definecolor{lightgray}{rgb}{.95,.95,.95}
\definecolor{darkgray}{rgb}{.4,.4,.4}
\definecolor{darkgreen}{rgb}{0,0.6,0}
\definecolor{pblue}{rgb}{0.13,0.13,1}
\definecolor{pgreen}{rgb}{0,0.5,0}
\definecolor{pred}{rgb}{0.9,0,0}
\definecolor{pgrey}{rgb}{0.46,0.45,0.48}

% Link color setup
\hypersetup{
	colorlinks,
	citecolor=purple,
	linkcolor=red,
	urlcolor=blue
}

% Custom header
\fancyhead[LO,RE]{}
\fancyhead[LE,RO]{}

% Table spacing
\setlength{\tabcolsep}{0.6em}
{\renewcommand{\arraystretch}{1.05}

% Math mode columns for tabular
\newcolumntype{L}{>{$}l<{$}}
\newcolumntype{R}{>{$}r<{$}}
\newcolumntype{C}{>{$}c<{$}}

% Fusion of characters
\makeatletter
\def\moverlay{\mathpalette\mov@rlay}
\def\mov@rlay#1#2{\leavevmode\vtop{%
   \baselineskip\z@skip \lineskiplimit-\maxdimen
   \ialign{\hfil$\m@th#1##$\hfil\cr#2\crcr}}}
\newcommand{\charfusion}[3][\mathord]{
    #1{\ifx#1\mathop\vphantom{#2}\fi
        \mathpalette\mov@rlay{#2\cr#3}
      }
    \ifx#1\mathop\expandafter\displaylimits\fi}
\makeatother

% Program language listings
\lstdefinelanguage{Java}{
	keywords={new, true, false, catch, return, null, catch, switch, var, if, in, while, do, else, case, break, public, private, package, import, final, protected, int, void, class},
	keywordstyle=\color{pblue}\bfseries,
	keywords=[2]{this},
	keywordstyle=[2]\color{darkgray}\bfseries,
	identifierstyle=\color{black},
	sensitive=false,
	comment=[l]{//},
	morecomment=[s]{/*}{*/},
	commentstyle=\color{pgreen}\ttfamily,
	stringstyle=\color{pred}\ttfamily,
	morestring=[b]',
	morestring=[b]"
}
\lstset{
	language=Java,
	backgroundcolor=\color{lightgray},
	extendedchars=true,
	basicstyle=\footnotesize\ttfamily,
	showstringspaces=false,
	showspaces=false,
	numbers=left,
	numberstyle=\footnotesize,
	numbersep=9pt,
	tabsize=2,
	breaklines=true,
	showtabs=false,
	captionpos=b
}
\lstdefinestyle{JavaStyle}{
	language=Java,
	frame=lines
}
\lstdefinelanguage{XML}{
	keywords={new, true, false, catch, return, null, catch, switch, var, if, in, while,
		do, else, case, break, public, private, package, import, final, protected, int,
		void, class, xml, osm, bounds, node, tag, way, nd, relation, member},
	keywordstyle=\color{pblue}\bfseries,
	keywords=[2]{this},
	keywordstyle=[2]\color{darkgray}\bfseries,
	identifierstyle=\color{black},
	sensitive=false,
	comment=[l]{//},
	morecomment=[s]{/*}{*/},
	commentstyle=\color{pgreen}\ttfamily,
	stringstyle=\color{pred}\ttfamily,
	morestring=[b]',
	morestring=[b]"
}
\lstset{
	language=XML,
	backgroundcolor=\color{lightgray},
	extendedchars=true,
	basicstyle=\footnotesize\ttfamily,
	showstringspaces=false,
	showspaces=false,
	numbers=left,
	numberstyle=\footnotesize,
	numbersep=9pt,
	tabsize=2,
	breaklines=true,
	showtabs=false,
	captionpos=b
}
\lstdefinestyle{XMLStyle}{
	language=XML,
	frame=lines
}
\lstdefinelanguage{Filter}{
	keywords={highway, way, area, train, access, type, railway, highway, building},
	keywordstyle=\color{pblue}\bfseries,
	keywords=[2]{KEEP, DROP},
	keywordstyle=[2]\color{pred}\bfseries,
	identifierstyle=\color{black},
	sensitive=false,
	comment=[l]{\#},
	morecomment=[s]{/*}{*/},
	commentstyle=\color{pgreen}\ttfamily,
	stringstyle=\color{pred}\ttfamily,
	morestring=[b]',
	morestring=[b]"
}
\lstset{
	language=Filter,
	backgroundcolor=\color{lightgray},
	extendedchars=true,
	basicstyle=\footnotesize\ttfamily,
	showstringspaces=false,
	showspaces=false,
	numbers=left,
	numberstyle=\footnotesize,
	numbersep=9pt,
	tabsize=2,
	breaklines=true,
	showtabs=false,
	captionpos=b
}
\lstdefinestyle{FilterStyle}{
	language=Filter,
	frame=lines
}
\lstdefinelanguage{GTFS}{
	keywords={agency_id, agency_name, agency_url, agency_timezone, agency_phone, agency_lang,
		stop_id, stop_name, stop_desc, stop_lat, stop_lon, stop_url, location_type, parent_station,
		route_id, route_short_name, route_long_name, route_desc, route_type,
		route_id, service_id, trip_id, trip_headsign, block_id,
		trip_id, arrival_time, departure_time, stop_id, stop_sequence, pickup_type, drop_off_type,
		service_id, monday, tuesday, wednesday, thursday, friday, saturday, sunday, start_date, end_date,
		service_id, date, exception_type,
		fare_id, price, currency_type, payment_method, transfers, transfer_duration,
		fare_id, route_id, origin_id, destination_id, contains_id,
		shape_id, shape_pt_lat, shape_pt_lon, shape_pt_sequence, shape_dist_traveled,
		trip_id, start_time, end_time, headway_secs,
		from_stop_id, to_stop_id, transfer_type, min_transfer_time},
	keywordstyle=\color{pblue}\bfseries,
	identifierstyle=\color{black},
	sensitive=false,
	comment=[l]{//},
	morecomment=[s]{/*}{*/},
	commentstyle=\color{pgreen}\ttfamily,
	stringstyle=\color{pred}\ttfamily,
	morestring=[b]',
	morestring=[b]"
}
\lstset{
	language=GTFS,
	backgroundcolor=\color{lightgray},
	extendedchars=true,
	basicstyle=\footnotesize\ttfamily,
	showstringspaces=false,
	showspaces=false,
	numbers=left,
	numberstyle=\footnotesize,
	numbersep=9pt,
	tabsize=2,
	breaklines=true,
	showtabs=false,
	captionpos=b
}
\lstdefinestyle{GTFSStyle}{
	language=GTFS,
	frame=lines,
}

% Custom commands
\newcommand{\zz}{$\mathrm{T\kern-.4em\raise-0.5ex\hbox{P}}$}
\newcommand\restr[2]{{\left.\kern-\nulldelimiterspace#1\vphantom{\big|}\right|_{#2}}}
\newcommand{\pto}{\mathrel{\ooalign{\hfil$\mapstochar$\hfil\cr$\to$\cr}}}
\newcommand{\ito}{\xrightarrow{\,\smash{\raisebox{-0.45ex}{\ensuremath{\scriptstyle\sim}}}\,}}
\newcommand{\cupdot}{\charfusion[\mathbin]{\cup}{\cdot}}
\newcommand{\bigcupdot}{\charfusion[\mathop]{\bigcup}{\cdot}}
\newcommand{\timef}[3]{\ensuremath{#1}\text{:}\ensuremath{#2}\textit{ #3}}
\newcommand{\figref}[1]{\textbf{Fig.\,\ref{#1}}}
\newcommand{\tableref}[1]{\textbf{Table \ref{#1}}}
\newcommand{\progref}[1]{\textbf{Prg.\,\ref{#1}}}
\newcommand{\lstref}[1]{\textbf{Lst.\,\ref{#1}}}
\newcommand{\lemmaref}[1]{\textbf{Lemma \ref{#1}}}
\newcommand{\theoremref}[1]{\textbf{Theorem \ref{#1}}}
\newcommand{\sectionref}[1]{\textbf{Section \ref{#1}}}
\newcommand{\fakesectionref}[1]{\textbf{Section #1}}
\newcommand{\algoref}[1]{\textbf{Algorithm \ref{#1}}}
\newcommand{\defref}[1]{\textbf{Definition \ref{#1}}}
\newcommand{\libref}[1]{\textbf{\cite{#1}}}
\newtheorem{mydef}{Definition}
\newtheorem{mylemma}{Lemma}
\newtheorem{mytheorem}{Theorem}
\newtheorem{mycorollary}{Colorally}
\newtheorem{myproposition}{Proposition}
\newtheorem{myclaim}{Claim}

% Prevent accidental usage
\renewcommand{\succ}{\error}
\renewcommand{\implies}{\error}

% Hacks
\newcommand{\todo}[1]{\fcolorbox{red}{yellow!30}{\textcolor{red}{TODO: #1}}}
\newcommand{\macrohighlight}[1]{\textcolor{magenta}{#1}}
\newcommand{\macro}[1]{\macrohighlight{#1}\xspace}
\newcommand{\macrosc}[1]{\macro{\textsc{#1}}}
\newcommand{\macromath}[1]{\macro{\ensuremath{#1}}}
\newcommand{\macromathsf}[1]{\macro{\ensuremath{\textsf{#1}}}}

% Disable hacks
%\renewcommand{\todo}[1]{}
%\renewcommand{\macrohighlight}[1]{#1}

% Macros
\newcommand{\java}{\macrosc{Java}}
\newcommand{\astar}{\macrosc{A\ensuremath{^{\star}}}}
\newcommand{\alt}{\macrosc{ALT}}
\newcommand{\bfs}{\macrosc{BFS}}
\newcommand{\csa}{\macrosc{CSA}}
\newcommand{\transferPatterns}{\macrosc{Transfer Patterns}}
\newcommand{\coverTree}{\macrosc{Cover Tree}}
\newcommand{\opnv}{\macrosc{ÖPNV}}
\newcommand{\dijkstra}{\macrosc{Dijkstra}}
\newcommand{\cobweb}{\macrosc{Cobweb}}
\newcommand{\nns}{\macrosc{NNS}}
\newcommand{\kdTree}{\macrosc{k-d tree}}
\newcommand{\vpTree}{\macrosc{VP tree}}
\newcommand{\bkTree}{\macrosc{BK-tree}}
\newcommand{\earliestArrivalProblem}{\macrosc{Earliest Arrival Problem}}
\newcommand{\shortestPathProblem}{\macrosc{Shortest Path Problem}}
\newcommand{\nearestNeighborProblem}{\macrosc{Nearest Neighbor Problem}}
\newcommand{\labelConstrainedShortestPathProblem}{\macrosc{Label-Constrained Shortest Path Problem}}
\newcommand{\dfa}{\macrosc{DFA}}
\newcommand{\lcspp}{\macrosc{LCSPP}}
\newcommand{\anr}{\macrosc{ANR}}
\newcommand{\accessNodeRouting}{\macrosc{Access-Node Routing}}
\newcommand{\osm}{\macrosc{OSM}}
\newcommand{\gtfs}{\macrosc{GTFS}}
\newcommand{\xml}{\macrosc{XML}}
\newcommand{\zip}{\macrosc{ZIP}}
\newcommand{\csv}{\macrosc{CSV}}

\newcommand{\true}{\macromathsf{true}}
\newcommand{\false}{\macromathsf{false}}
\newcommand{\vmax}{\macromathsf{max}}
\newcommand{\vmin}{\macromathsf{min}}
\newcommand{\vundef}{\macromathsf{undefined}}
\newcommand{\car}{\macromathsf{car}}
\newcommand{\bike}{\macromathsf{bike}}
\newcommand{\foot}{\macromathsf{foot}}
\newcommand{\tram}{\macromathsf{tram}}

\newcommand{\inlineCode}[1]{{\lstinline[style={JavaStyle}]{#1}}}

\newcommand{\arr}{\macromathsf{arr}}
\newcommand{\dep}{\macromathsf{dep}}
\newcommand{\arrival}{\macromathsf{arrival}}
\newcommand{\departure}{\macromathsf{departure}}
\newcommand{\transfer}{\macromathsf{transfer}}
\newcommand{\freiburg}{\macromathsf{Freiburg Hbf}}
\newcommand{\offenburg}{\macromathsf{Offenburg}}
\newcommand{\karlsruhe}{\macromathsf{Karlsruhe Hbf}}
\newcommand{\ticef}{\macromathsf{ICE 104}}
\newcommand{\tregiof}{\macromathsf{RE 17024}}
\newcommand{\tregios}{\macromathsf{RE 17322}}
\newcommand{\tices}{\macromathsf{ICE 79}}
\newcommand{\enter}{\macromathsf{enter}}
\newcommand{\exit}{\macromathsf{exit}}
\newcommand{\uniModal}{\macromathsf{uni-modal}}
\newcommand{\multiModal}{\macromathsf{multi-modal}}
\newcommand{\freiburgR}{\macromathsf{Freiburg}}
\newcommand{\stuttgartR}{\macromathsf{Stuttgart}}
\newcommand{\switzerlandR}{\macromathsf{Switzerland}}

% Operator
\DeclareMathOperator*{\argmax}{arg\,max}
\DeclareMathOperator*{\argmin}{arg\,min}
\DeclareMathOperator{\asTheCrowFlies}{asTheCrowFlies}
\DeclareMathOperator{\link}{link}
\DeclareMathOperator{\src}{src}
\DeclareMathOperator{\dest}{dest}
\DeclareMathOperator{\depth}{depth}
\DeclareMathOperator{\height}{height}
\DeclareMathOperator{\lvl}{lvl}
\DeclareMathOperator{\assoc}{assoc}
\DeclareMathOperator{\children}{children}
\DeclareMathOperator{\dist}{dist}
\DeclareMathOperator{\orev}{prev}
\DeclareMathOperator{\landmarks}{landmarks}
\DeclareMathOperator{\mode}{mode}
\DeclareMathOperator{\degG}{deg}

% Document
\begin{document}
% Command renews that need to be inside the document
\renewcommand{\figurename}{Fig.}% Fig. instead of Figure when linking figures

% Begin of content
% Titlepage
\begin{titlepage}
	\hypersetup{urlcolor=black}
	\title{Multi-Modal Route Planning in Road and Transit Networks}
	\subtitle{Master's Thesis} 
	\author{Daniel Tischner\\\quad\\
		\small{University of Freiburg, Germany,}\\
		\small{\href{mailto:daniel.tischner.cs@gmail.com}{\texttt{daniel.tischner.cs@gmail.com}}}\\\quad}
	\publishers{
		\begin{tabular}{ll}
			\\
			Supervisor:	& Prof.~Dr.~Hannah Bast\\
			Advisor:		&  Patrick Brosi
		\end{tabular}
		\todo{disable todos and macro highlights.}% Remove todo for final submission
	}
	\maketitle
	% Suppress page number on title page
	\thispagestyle{empty}
\end{titlepage}
\newpage
{
	\hypersetup{linkcolor=black}
	\tableofcontents
}
\clearpage
% Declaration
\renewcommand{\abstractname}{\huge Declaration}
\begin{abstract}
	\vbox{}
	I hereby declare, that I am the sole author and composer of my Thesis and that
	no other sources or learning aids, other than those listed, have been used.
	Furthermore, I declare that I have acknowledged the work of others by providing
	detailed references of said work. I hereby also declare, that my Thesis has not
	been prepared for another examination or assignment, either
	wholly or excerpts thereof.
	\vfill
	\parbox{4cm}{\hrule
	\strut \centering\footnotesize Place, Date} \hfill\parbox{4cm}{\hrule
	\strut \centering\footnotesize Signature}
\end{abstract}
\clearpage

\renewcommand{\abstractname}{\huge Zusammenfassung}
\begin{abstract}
	\vbox{}
	Wir präsentieren Algorithmen für {\multiModal}e Routenplanung
	in Straßennetzwerken und Netzwerken des öffentlichen
	Personennahverkehrs (\opnv), so wie in kombinierten Netzwerken.
	
	Dazu stellen wir das Nächste-Nachbar- und das Kürzester-Pfad-Problem
	vor und schlagen Lösungen basierend auf {\coverTree}s, \alt und \csa vor.
	
	Des Weiteren erläutern wir die Theorie hinter den Algorithmen, geben eine
	kurze Übersicht über andere Techniken, zeigen Versuchsergebnisse auf und
	vergleichen die Techniken untereinander.
	\vfill
\end{abstract}
\clearpage
\renewcommand{\abstractname}{\huge Abstract}
\begin{abstract}\quad\\
	We present algorithms for \multiModal route planning
	in road and public transit networks, as well as in combined networks.
	
	Therefore, we explore the nearest neighbor and shortest path problem
	and propose solutions based on {\coverTree}s, \alt and \csa.
	
	Further, we illustrate the theory behind the algorithms, give a short
	overview of other techniques, present experimental results and compare
	the techniques with each other.
\end{abstract}
\clearpage
% Commands that need to be after the preamble
\fancyhead[LO,RE]{\parbox{0.7\textwidth}{\textbf{\rightmark}}}
\fancyhead[LE,RO]{\textbf{Section \arabic{section}}}

% Introduction
\section{Introduction}\label{introduction}
	Route planning refers to the problem of finding an \textit{optimal} route between given locations in a network.
	With the ongoing expansion of road and public transit networks all over the world route planner gain more and
	more importance. This led to a rapid increase in research \libref{routePlanningOverview, networks, transitModels}
	of relevant topics and development of route planner software \libref{navHistoryEarly, navHistoryNewer, vehicleNavigation}.
	
	However, a common problem of most such services is that they are limited to one transportation mode only.
	That is a route can only be taken by a car or train but not by both at the same time. This is known as uni-modal routing.
	In contrast to that multi-modal routing allows the alternation of transportation modes. For example a route that
	first uses a car to drive to a train station, then a train which travels to a another train station and finally
	using a bicycle from there to reach the destination.
	
	The difficulty with multi-modal routing lies in most algorithms being fitted to networks with specific properties.
	Unfortunately, road networks differ a lot from public transit networks. As such, a route planning algorithm
	fitted to a certain type of network will likely yield undesired results, have an impractical running time or not
	even be able to be used at all on different networks. We will explore this later in \sectionref{evaluation}.\\\\
	In this thesis we explore a technique with which we can combine an algorithm fitted for road networks with an algorithm
	for public transit networks. Effectively obtaining a generic algorithm that is able to compute routes on combined networks.
	The basic idea is simple, given a source and destination, both in the road network, we select \textit{access nodes} for both.
	This are nodes where we will switch from the road into the public transit network. A route can then be computed by
	using the road algorithm for the source to its access nodes, the transit algorithm for the access nodes of the source
	to the access nodes of the destination and finally the road algorithm again for the destinations access nodes to
	the destination. Note that this technique might not yield the shortest possible path anymore. Also, it does not allow
	an arbitrary alternation of transportation modes. However, we accept those limitations since the resulting
	algorithm is very generic and able to compute routes faster than without limitations. We will cover this technique in detail
	in \sectionref{accessNodes}.\\\\
	Our final technique uses a modified version of \alt \libref{alt} as road algorithm and \csa \libref{csa} for the transportation network.
	The algorithms are presented in \sectionref{alt} and \sectionref{csa} respectively.
	We also develop a multi-modal variant of \dijkstra \libref{dijkstra} which is able to compute the shortest route in a combined
	network with the possibility of changing transportation modes arbitrarily. It is presented in \sectionref{modifiedDijkstra}
	and acts as baseline to our final technique based on access nodes.
	
	We compute access nodes by solving the \nearestNeighborProblem. For a given node in the road network its access
	nodes are then all nodes in the transit network which are in the \textit{vicinity} of the road node. We explore a solution
	to this problem in \sectionref{nearestNeighborProblem}.\\\\
	\sectionref{models} starts by defining types of networks. We represent road networks by graphs only.
	For transit networks we provide a graph representation too. Both graphs can then be combined into a linked graph.
	The advantage of graph based models is that they are well studied and therefore we are able to use our
	multi-modal variant of \dijkstra to compute routes on them.
	However, we also propose a non-graph based representation for transit networks, a timetable. The timetable is used by \csa,
	an efficient algorithm for route planning on public transit networks. With that, our road and transit networks get incompatible
	and can not easily be combined. Therefore, we use the previously mentioned generic approach based on access nodes
	for this type of network.\\\\
	% Cobweb frontend screenshot
	\begin{figure}[!ht]
		 \begin{center}
			\includegraphics[scale=0.5]{res/cobweb_frontend}
		\end{center}
		\caption{Screenshot of {\cobweb}s \libref{cobweb} frontend, an open-source multi-modal route planner. It shows a multi-modal
		route starting from a given source, using the modes \textit{foot-tram-foot-tram-foot} in that sequence to reach the destination.}
		\label{cobweb_frontend}
	\end{figure}\quad\\
	Further, we implemented the presented algorithms in the \cobweb \libref{cobweb} project, which is an open-source multi-modal
	route planner (see \figref{cobweb_frontend} for an image of its frontend).
	In \sectionref{evaluation} we show our experimental results and compare the techniques with each other.
% Preliminaries
\section{Preliminaries}\label{preliminaries}
	Before we define our specific data models and problems we will introduce and formalize commonly reoccurring terms.

% Graph
\subsection{Graph}
	\begin{mydef}\label{graph}
		A \textnormal{graph} $G$ is a tuple $(V, E)$ with a set of nodes $V$ and a set of
		edges $E \subseteq V \times \mathbb{R}_{\ge 0} \times V$.
		An \textnormal{edge} $e \in E$ is an ordered tuple $(u, w, v)$ with source node $u \in V$, a non-negative
		weight $w \in \mathbb{R}_{\ge 0}$ and a destination node $v \in V$.
	\end{mydef}\quad\\
	Note that \defref{graph} actually defines a \textit{directed} graph, as opposed to an \textit{undirected} graph where an
	edge like $(u, w, v)$ would be considered equal to the edge of opposite direction $(v, w, u)$ (compare to \libref{graphTheory}).
	However, for transportation networks an undirected graph often is not applicable, for example due to one way streets or
	time dependent connections like trains which depart at different times for different directions.
	
	In the context of route planning we refer to the weight $w$ of an edge $(u, w, v)$ as \textit{cost}. It can be used to encode the length
	of the represented connection. Or to represent the time it takes to travel the distance in a given
	transportation mode.
	% Graph example
	\begin{figure}[!ht]
		\begin{center}
			\begin{tikzpicture}[y = -1cm]
			 	% Nodes
			 	\node[circle, draw] (v1) at (0, 0) {$v_1$};
			 	\node[circle, draw] (v2) at (2, 0) {$v_2$};
			 	\node[circle, draw] (v3) at (0, 2) {$v_3$};
			 	\node[circle, draw] (v4) at (2, 2) {$v_4$};
			 	\node[circle, draw] (v5) at (4, 0) {$v_5$};
			 	
			 	% Edges
			 	\draw[thick, ->] (v1) to [bend left] node[above] {$8$} (v2);
			 	\draw[thick, ->] (v2) to node[above] {$2$} (v5);
			 	\draw[thick, ->] (v2) to [bend left] node[below] {$1$} (v1);
			 	\draw[thick, ->] (v1) to node[left] {$1$} (v3);
			 	\draw[thick, ->] (v3) to node[above] {$2$} (v4);
			 	\draw[thick, ->] (v4) to node[left] {$1$} (v2);
			\end{tikzpicture}
		\end{center}
		\caption{Illustration of an example graph with five nodes and six edges.}
		\label{graph_example}
	\end{figure}\quad\\
	As an example consider the graph $G = (V, E)$ with
	\begin{align*}
		V	&= \{v_1, v_2, v_3, v_4, v_5\} \text{ and}\\
		E	&= \{(v_1, 8, v_2), (v_1, 1, v_3), (v_2, 1, v_1), (v_2, 2, v_5), (v_3, 2, v_4), (v_4, 1, v_2)\}.
	\end{align*}
	which is illustrated by \figref{graph_example}.
	\begin{mydef}
		Given a graph $G = (V, E)$ the function $\src: E \to V, ((u, w, v)) \mapsto u$ gets the \textnormal{source}
		of an edge. Analogously $\dest: E \to V, ((u, w, v)) \mapsto v$ retrieves the \textnormal{destination}.
	\end{mydef}
	\begin{mydef}\label{path}
		A \textnormal{path} in a graph $G = (V, E)$ is a sequence $p = e_1e_2e_3\ldots$ of edges $e_i \in E$ such that
		\begin{align*}
			\forall i: \dest(e_i) = \src(e_{i + 1}).
		\end{align*}
		We write $e \in p$ if an edge $e$ is contained at least once in the path $p$.
		The \textnormal{length} of a path is the amount of edges it contains, i.e. the length of the sequence.
		The \textnormal{weight} or \textnormal{cost} is the sum of its edges weights.
		
		Let $k$ be the length of a path $p$, then we define:
		\begin{align*}
			\src(p)	&= \src(e_1)\\
			\dest(p)	&= \dest(e_k)
		\end{align*}\quad\\
		Given two paths $q_1 = e_1\ldots e_k$ and $q_2 = e'_1\ldots e'_l$ where $\dest(e_k) = \src(e'_1)$,
		the concatenation of both paths is a path
		\begin{align*}
			p	&= e_1\ldots e_k e'_1\ldots e'_l
		\end{align*}
		with length $k + l$, also denoted by $p = q_1q_2$.
	\end{mydef}\quad\\
	An example for a path in the graph $G$ would be
	\begin{align*}
		p	&=(v_1, 8, v_2)(v_2, 1, v_1)(v_1, 1, v_3).
	\end{align*}
	Its length is $3$ and it has a weight of $10$.
	
% Tree
\subsection{Tree}
	\begin{mydef}\label{tree}
		A \textnormal{tree} is an graph $T = (V, E)$ with the following properties:
		\begin{itemize}
			\item[1.] There is exactly one node $r \in V$ with no ingoing edges, called the \textnormal{root}, i.e.
				\begin{align*}
					\exists! r \in V \nexists e \in E : \dest(e) = r.
				\end{align*}
			\item[2.] All other nodes $v$ have exactly one ingoing edge. The source $p$ of this edge is called \textnormal{parent} of $v$ and
				$v$ is called \textnormal{child} of $p$:
				\begin{align*}
					\forall v \in V : v \neq r \Rightarrow \exists! e \in E : \dest(e) = v.
				\end{align*}
		\end{itemize}
	\end{mydef}
	\begin{mydef}\label{subTree}
		The \textnormal{subtree} of a tree $T = (V, E)$ rooted at a node $r' \in V$ is a tree $T' = (V', E')$. $V' \subseteq V$ is the set
		of nodes that can be reached from $r'$. That is, all nodes that are part of possible paths starting at $r'$.
		Likewise $E' \subseteq E$ is the set of edges restricted to the vertices in $V'$. The root of $T'$ is $r'$.
	\end{mydef}
	\begin{mydef}\label{treeDepth}
		The \textnormal{depth} of a node $v$ in a tree $T = (V, E)$, denoted by $\depth(v)$, is defined as the amount of
		edges between the $v$ and the root $r$. It is the length of the unique path $p$ starting at $r$ and ending at $v$.
		
		The \textnormal{height} of a tree is its greatest depth, i.e.
		\begin{align*}
			\max_{v \in V} \depth(v).
		\end{align*}
		And
		\begin{align*}
			\children(v)	&= \{c \in T | c \text{ child of } v\}.
		\end{align*}
	\end{mydef}\quad\\
	Trees are hierarchical data structures. Every node, except the root, has one parent. A node itself can have multiple children.
	Note that it is not possible to form a loop in a tree, i.e. a path that visits a node more than once. A node without
	children is called a \textit{leaf}.\\
	% Tree example
	\begin{figure}[!ht]
		\begin{center}
			\begin{tikzpicture}[y = -1cm]
			 	% Nodes
			 	\node[circle, draw] (v1) at (3, 0) {$v_1$};
			 	\node[circle, draw] (v2) at (1, 1.5) {$v_2$};
			 	\node[circle, draw] (v3) at (3, 1.5) {$v_3$};
			 	\node[circle, draw] (v4) at (4, 1.5) {$v_4$};
			 	\node[circle, draw] (v5) at (0, 3) {$v_5$};
			 	\node[circle, draw] (v6) at (2, 3) {$v_6$};
			 	\node[circle, draw] (v7) at (4, 3) {$v_7$};
			 	
			 	% Edges
			 	\draw[thick, ->] (v1) to (v2);
			 	\draw[thick, ->] (v1) to (v3);
			 	\draw[thick, ->] (v1) to (v4);
			 	\draw[thick, ->] (v2) to (v5);
			 	\draw[thick, ->] (v2) to (v6);
			 	\draw[thick, ->] (v4) to (v7);
			\end{tikzpicture}\qquad\qquad\qquad
			\begin{tikzpicture}[y = -1cm]
			 	% Nodes
			 	\node[circle, draw] (v2) at (1, 0) {$v_2$};
			 	\node[circle, draw] (v5) at (0, 1.5) {$v_5$};
			 	\node[circle, draw] (v6) at (2, 1.5) {$v_6$};
			 	
			 	% Edges
			 	\draw[thick, ->] (v2) to (v5);
			 	\draw[thick, ->] (v2) to (v6);
			\end{tikzpicture}
		\end{center}
		\caption{Example of an unlabeled tree (left) and the subtree of $v_2$ (right).}
		\label{treeExample}
	\end{figure}\quad\\
	\figref{treeExample} shows a tree with $7$ nodes. The node $v_1$ is the root; $v_5, v_6, v_3$
	and $v_7$ are the leaves. The tree has a height of $2$, the depth of $v_4$ is $1$. The subtree rooted at $v_2$
	only consists of the nodes $v_2, v_5$ and $v_6$.

% Automaton
\subsection{Automaton}\label{automaton_sec}
	Automata are labeled graphs. They are used to represent states and the correlation between them.
	\begin{mydef}\label{automaton}
		A \textnormal{deterministic finite automaton} (\dfa) $A$ is a tuple $(Q, \sigma, \Delta, q_0, F)$ with
		\begin{itemize}
			\item[] a set of states $Q$,
			\item [] a set of labels $\sigma$, called \textnormal{alphabet},
			\item[] a transition relation $\Delta \subseteq Q \times \sigma \times Q$,
			\item[] an initial state $q_0 \in Q$ and
			\item[] a set of accepting states $F \subseteq Q$.
		\end{itemize}
	\end{mydef}
	\begin{mydef}
		A \textnormal{word} $w \in \Sigma^{\star}$ is a finite sequence of letters
		\begin{align*}
			w	&= a_0a_1a_2 \ldots a_{k - 1}
		\end{align*}
		with $a_i \in \Sigma$ and some $k \in \mathbb{N}$. The empty word is denoted by $\varepsilon$.
		
		A word is called \textnormal{accepted} iff
		\begin{itemize}
			\item[1.] \begin{align*}
					\forall i: (q_i, a_i, q_{i + 1}) \in \Delta,
				\end{align*}
				for some $q_i \in Q$,
			\item[2.] $q_0$ is the initial state of the automaton and
			\item[3.] the last state is accepting, i.e. $q_k \in F$.
		\end{itemize}
		We say, the automaton $A$ accepts the word $w$.
	\end{mydef}
	\begin{mydef}
		The language $\mathcal{L}(A)$ of an automaton $A$ is defined as the set of accepted words:
		\begin{align*}
			\mathcal{L}(A)	&= \{w \in \Sigma^{\star} | A \text{ accepts } w\}
		\end{align*}
	\end{mydef}\quad\\
	% Automaton example
	\begin{figure}[!ht]
		\begin{center}
			\begin{tikzpicture}[y = -1cm]
			 	% Nodes
			 	\node[initial, state] (q0) at (0, 0) {$q_0$};
			 	\node[state] (q1) at (2, 0) {$q_1$};
			 	\node[accepting, state] (q2) at (4, 0) {$q_2$};
			 	
			 	% Edges
			 	\draw[thick, ->] (-1, 0) to (q0);
			 	\draw[thick, ->] (q0) to [bend left] node[above] {$a$} (q1);
			 	\draw[thick, ->] (q1) to [bend left] node[below] {$b$} (q0);
			 	\draw[thick, ->] (q1) to node[above] {$c$} (q2);
			\end{tikzpicture}
		\end{center}
		\caption{Example of a deterministic finite automaton. $q_0$ is the initial state and $q_2$ is accepting.}
		\label{automatonExample}
	\end{figure}\quad\\
	For an example, refer to \figref{automatonExample} which accepts the language
	\begin{align*}
		(ab)^{\star}ac
	\end{align*}
	denoting words with a finite sequence of $ab$, then one $a$ and one $c$. Such as:
	\begin{align*}
		&ac\\
		&abac\\
		&ababac\\
		&abababac\\
		&\vdots
	\end{align*}

% Metric
\subsection{Metric}
	\begin{mydef}\label{metric}
		A function $d: M \times M \to \mathbb{R}$ on a set $M$ is called a \textnormal{metric} iff for all $x, y, z \in M$
		\begin{align*}
			d(x, y)	&\ge 0,			&&\text{non-negativity}\\
			d(x, y) = 0	&\Leftrightarrow x = y,	&&\text{identity of indiscernibles}\\
			d(x, y)	&= d(y, x) \text{ and }	&&\text{symmetry}\\
			d(x, z)	&\le d(x, y) + d(y, z)	&&\text{triangle inequality}
		\end{align*}
		holds.
	\end{mydef}
	\begin{mydef}\label{metricSpace}
		A \textnormal{metric space} is a pair $(M, d)$ where $M$ is a set
		and $d: M \times M \to \mathbb{R}$ a metric on $M$.
	\end{mydef}
	\begin{mydef}\label{metricSet}
		Given a metric $d$ on a set $M$, the distance of a point $p \in M$ to a subset $Q \subseteq M$
		is defined as the distance from $p$ to its nearest point in $Q$:
		\begin{align*}
			d(p, Q)	&= \min_{q \in Q} d(p, q)
		\end{align*}
	\end{mydef}\quad\\
	A metric is used to measure the distance between given locations. \sectionref{nearestNeighborProblem}
	and \sectionref{shortestPathProblem}, in particular \sectionref{alt}, will make heavy use of this term.
	
	There we measure the distance between geographical locations given as pair of \textit{latitude} and \textit{longitude} coordinates.
	Latitude and longitude, often denoted by $\phi$ and $\lambda$, are real numbers in the ranges $(-90, 90)$ and $[-180, 180)$ respectively,
	measured in degrees. However, for convenience we represent them in radians. Both representations are equivalent to each other
	and can easily be converted using the ratio $360^\circ = 2 \pi \text{ rad}$.\\\\
	A commonly used measure is the \textit{as-the-crow-flies} metric, which is equivalent to the euclidean distance in the euclidean space.
	\defref{asTheCrowFlies} defines an approximation of this distance on locations given by latitude and longitude coordinates.
	The approximation is commonly known as equirectangular projection of the earth \libref{equiRectProjection}.
	Note that there are more accurate methods for computing the great-circle distance for geographical locations,
	like the haversine formula \libref{haversine}. However, they come with a significant computational overhead.
	\begin{mydef}\label{asTheCrowFlies}
		Given a set of coordinates $M = \left\{(\phi, \lambda) | \phi \in \left(-\frac{\pi}{2}, \frac{\pi}{2}\right), \lambda \in [-\pi, \pi)\right\}$ we define
		$\asTheCrowFlies: M \times M \to \mathbb{R}$ such that
		\begin{align*}
			\left(\left(\phi_1, \lambda_1\right), \left(\phi_2, \lambda_2\right)\right) \mapsto
				\sqrt{\left(\left(\lambda_2 - \lambda_1\right) \cdot \cos\left(\frac{\phi_1 + \phi_2}{2}\right)\right)^2
					+ \left(\phi_2 - \phi_1\right)^2} \cdot 6371000.
		\end{align*}
	\end{mydef}\quad\\
	As a next step we prove that $\asTheCrowFlies$ is indeed a metric on the set of coordinates.
	\begin{myproposition}
		The function $\asTheCrowFlies$ is a metric on its domain $M$.
	\end{myproposition}
	\begin{proof}
		We need to prove that all four axioms hold. Let us first set
		\begin{align*}
			x	&= \left(\lambda_2 - \lambda_1\right) \cdot \cos\left(\frac{\phi_1 + \phi_2}{2}\right)\\
			y	&= \phi_2 - \phi_1
		\end{align*}
		then the function simplifies to
		\begin{align*}
			\sqrt{x^2 + y^2} \cdot 6371000.
		\end{align*}
		Obviously this can never resolve to a negative number since
		\begin{align*}
			\underbrace{\underbrace{\sqrt{\underbrace{x^2}_{\ge 0} + \underbrace{y^2}_{\ge 0}}}_{\ge 0} \cdot 6371000}_{\ge 0}.
		\end{align*}
		
		For the second axiom we assume that $\asTheCrowFlies\left(\left(\phi_1, \lambda_1\right), \left(\phi_2, \lambda_2\right)\right) = 0$
		for an arbitrary pair of coordinates and follow\\
		\begin{center}
			\begin{tabular}{RRL}
							&\sqrt{\left(\left(\lambda_2 - \lambda_1\right) \cdot \cos\left(\frac{\phi_1 + \phi_2}{2}\right)\right)^2
								+ \left(\phi_2 - \phi_1\right)^2} \cdot 6371000	&= 0\\
				\Leftrightarrow	&\sqrt{\left(\left(\lambda_2 - \lambda_1\right) \cdot \cos\left(\frac{\phi_1 + \phi_2}{2}\right)\right)^2
								+ \left(\phi_2 - \phi_1\right)^2} 				&= 0\\
				\Leftrightarrow	&\left(\left(\lambda_2 - \lambda_1\right) \cdot \cos\left(\frac{\phi_1 + \phi_2}{2}\right)\right)^2
								+ \left(\phi_2 - \phi_1\right)^2 				&= 0
			\end{tabular}
		\end{center}
		At this point either both summands are $0$ or one is the negative of the other. However, both summands must be positive due to the quadration.
		Because of that we follow\\
		\begin{center}
			\begin{tabular}{RRL}
							&\left(\phi_2 - \phi_1\right)^2 	&= 0\\
				\Leftrightarrow	&\phi_2				&= \phi_1
			\end{tabular}
		\end{center}
		and with that\\
		\begin{center}
			\begin{tabular}{RRL}
							&\left(\left(\lambda_2 - \lambda_1\right) \cdot \cos\left(\frac{\phi_1 + \phi_2}{2}\right)\right)^2	&= 0\\
				\Leftrightarrow	&\left(\lambda_2 - \lambda_1\right) \cdot \cos\left(\frac{2 \phi_1}{2}\right)	&= 0\\
				\Leftrightarrow	&\left(\lambda_2 - \lambda_1\right) \cdot \cos\left(\phi_1\right)	&= 0.
			\end{tabular}
		\end{center}
		Since $\phi_1 \in \left(-\frac{\pi}{2}, \frac{\pi}{2}\right)$ it follows that $\cos\left(\phi_1\right) \ne 0$. As such
		\begin{center}
			\begin{tabular}{RRL}
							&\lambda_2 - \lambda_1	&= 0\\
				\Leftrightarrow	&\lambda_2			&= \lambda_1
			\end{tabular}
		\end{center}
		and by that $(\phi_1, \lambda_1) = (\phi_2, \lambda_2)$, so the second axiom holds.
		
		Symmetry follows quickly since
		\begin{align*}
			\phi_1 + \phi_2	&= \phi_2 + \phi_1\\
			(\phi_2 - \phi_1)^2	&= (\phi_1 - \phi_2)^2\\
			\left((\lambda_2 - \lambda_1) \cdot \cos\left(\frac{\phi_1 + \phi_2}{2}\right)\right)^2
				&= (\lambda_2 - \lambda_1)^2 \cdot \cos^2\left(\frac{\phi_1 + \phi_2}{2}\right)\\
			(\lambda_2 - \lambda_1)^2	&= (\lambda_1 - \lambda_2)^2.
		\end{align*}
		
		The triangle inequality is a bit trickier, we choose three arbitrary coordinates $c_i = (\phi_i, \lambda_i)$ for $i = 1, 2, 3$
		and start on the squared left side:
		\begin{align*}
			\asTheCrowFlies^2(c_1, c_3)
				&= \left(\left(\left(\lambda_3 - \lambda_1\right) \cdot \cos\left(\frac{\phi_1 + \phi_3}{2}\right)\right)^2
					+ \left(\phi_3 - \phi_1\right)^2\right) \cdot 6371000^2\\
				&= \left(\left(\left(\lambda_3 - \lambda_2 + \lambda_2 - \lambda_1\right) \cdot \cos\left(\frac{\phi_1 + \phi_3}{2}\right)\right)^2
					+ \left(\phi_3 - \phi_2 + \phi_2 - \phi_1\right)^2\right) \cdot 6371000^2\\
				&= \Bigg(\left((\lambda_3 - \lambda_2)^2 + (\lambda_2 - \lambda_1)^2 + 2 \cdot ((\lambda_3 - \lambda_2) \cdot (\lambda_2 - \lambda_1))\right)
						\cdot \cos^2\left(\frac{\phi_1 + \phi_3}{2}\right)\\
					&\qquad + (\phi_3 - \phi_2)^2 + (\phi_2 - \phi_1)^2 + 2 \cdot ((\pi_3 - \phi_2) \cdot (\phi_2 - \phi_1))\Bigg) \cdot 6371000^2\\
				&= ...\\
				&\le \left(\left(\left(\left(\lambda_2 - \lambda_1\right) \cdot \cos\left(\frac{\phi_1 + \phi_2}{2}\right)\right)^2
						+ \left(\phi_2 - \phi_1\right)^2\right) \cdot 6371000^2\right)\\
					&\qquad+ \left(\left(\left(\left(\lambda_3 - \lambda_2\right) \cdot \cos\left(\frac{\phi_2 + \phi_3}{2}\right)\right)^2
						+ \left(\phi_3 - \phi_2\right)^2\right) \cdot 6371000^2\right)\\
					&\qquad + 2 \cdot \left(\left(\left(\left(\lambda_2 - \lambda_1\right) \cdot \cos\left(\frac{\phi_1 + \phi_2}{2}\right)\right)^2
						+ \left(\phi_2 - \phi_1\right)^2\right) \cdot 6371000^2\right)\\
					&\qquad \cdot \left(\left(\left(\left(\lambda_3 - \lambda_2\right) \cdot \cos\left(\frac{\phi_2 + \phi_3}{2}\right)\right)^2
						+ \left(\phi_3 - \phi_2\right)^2\right) \cdot 6371000^2\right)\\
				&= \left(\asTheCrowFlies(c_1, c_2) + \asTheCrowFlies(c_2, c_3)\right)^2
		\end{align*}
		\todo{continue (squared ineq holds without, cauchy schwarz ineq) or remove completely...}
		All four axioms hold, $\asTheCrowFlies$ is a metric on the set $M$.
	\end{proof}
% Models
\section{Models}\label{models}
	Blabla

%Road graph
\subsection{Road graph}
	blabla

%Transit graph
\subsection{Transit graph}
	blabla

%Link graph
\subsection{Link graph}
	blabla

%Timetable
\subsection{Timetable}
	blabla
%Nearest neighbor problem
\chapter{Nearest neighbor problem}\label{nearestNeighborProblem}
	In this section we introduce the \nearestNeighborProblem, also known as nearest neighbor search (\nns).
	First, we define the problem. Then a short overview of related research is given, after which we elaborate on
	a solution called {\coverTree} \libref{coverTree}.
	\begin{mydef}
		Given a metric space $(M, d)$ (see \defref{metricSpace}) with $|M| \ge 2$ and a point $x \in M$,
		the \textnormal{nearest neighbor problem} asks for finding a point $y \in M$ such that
		\begin{align*}
			y = \argmin_{y' \in M \setminus \{x\}} d(x, y').
		\end{align*}
		The point $y$ is called \textnormal{nearest neighbor} of $x$.
	\end{mydef}
	% Nearest neighbor problem example
	\begin{figure}[!ht]
		 \begin{center}
			\begin{tikzpicture}[scale=0.75]
				% Grid
				\foreach \i [evaluate=\i as \ii using int(\i*10)] in {0,...,9} {
					\draw [very thin,gray] (\i,0) -- (\i,9)  node [below] at (\i,0) {\small{\color{black}{$\ii$}}};
					\draw [very thin,gray] (0,\i) -- (9,\i) node [left] at (0,\i) {\small{\color{black}{$\ii$}}};
				}
				
			 	% Nodes
			 	\node[inner sep=2pt, fill=black, circle, draw] (x1) at (5,5) {};
			 	\node[below right] at (x1) {$x_1$};
			 	
			 	\node[inner sep=2pt, fill=black, circle, draw] (x2) at (3,3) {};
			 	\node[below right] at (x2) {$x_2$};
			 	
			 	\node[inner sep=2pt, fill=black, circle, draw] (x3) at (3,7) {};
			 	\node[below right] at (x3) {$x_3$};
			 	
			 	\node[inner sep=2pt, fill=black, circle, draw] (x4) at (7,3) {};
			 	\node[below right] at (x4) {$x_4$};
			 	
			 	\node[inner sep=2pt, fill=black, circle, draw] (x5) at (7,7) {};
			 	\node[below right] at (x5) {$x_5$};
			 	
			 	\node[inner sep=2pt, fill=black, circle, draw] (x6) at (3,1.5) {};
			 	\node[below right] at (x6) {$x_6$};
			 	
			 	\node[inner sep=2pt, fill=black, circle, draw] (x7) at (2,3) {};
			 	\node[below right] at (x7) {$x_7$};
			 	
			 	\node[inner sep=2pt, fill=black, circle, draw] (x8) at (7,1.5) {};
			 	\node[below right] at (x8) {$x_8$};
			 	
			 	\node[inner sep=2pt, fill=black, circle, draw] (x9) at (8.5,3) {};
			 	\node[below right] at (x9) {$x_9$};
			 	
			 	\node[inner sep=2pt, fill=black, circle, draw] (x10) at (2,7) {};
			 	\node[below right] at (x10) {$x_{10}$};
			 	
			 	\node[inner sep=2pt, fill=black, circle, draw] (x11) at (1,8) {};
			 	\node[below right] at (x11) {$x_{11}$};
			\end{tikzpicture}
		\end{center}
		\caption{Grid showing eleven points in the cartesian plane $\mathbb{R}^2$.}
		\label{nearestNeighborProblemExample}
	\end{figure}\quad\\
	For following examples the toy data set shown in \figref{nearestNeighborProblemExample} is introduced.
	It consists of the points
	\begin{align*}
		x_1		&= (50, 50),\\
		x_2		&= (30, 30),\\
		x_3		&= (30, 70),\\
		x_4		&= (70, 30),\\
		x_5		&= (70, 70),\\
		x_6		&= (30, 15),\\
		x_7		&= (20, 30),\\
		x_8		&= (70, 15),\\
		x_9		&= (85, 30),\\
		x_{10}	&= (20, 70),\\
		x_{11}	&= (10, 80).
	\end{align*}
	All points are elements of the cartesian plane $\mathbb{R}$. The Euclidean distance $d$ is chosen as metric on this set.
	For two dimensions it can be defined as:
	\begin{align*}
		d: \mathbb{R}^2 \times \mathbb{R}^2, ((x_1, y_1), (x_2, y_2)) \mapsto \sqrt{(x_2 - x_1)^2 + (y_2 - y_1)^2}
	\end{align*}
	Informally $d$ computes the \textit{ordinary} straight-line distance between two points.\\\\
	The nearest neighbor of $x_5$ is $x_1$, as
	\begin{align*}
		d(x_5, x_1)	&= \sqrt{(50 - 70)^2 + (50 - 70)^2}\\
				&= \sqrt{800}
	\end{align*}
	is smaller than all other distances to $x_5$, like
	\begin{align*}
		d(x_5, x_4)	&= \sqrt{(70 - 70)^2 + (30 - 70)^2}\\
				&= \sqrt{1600}.
	\end{align*}
	On the other hand, $x_1$ has four smallest neighbors:
	\begin{align*}
		d(x_1, x_2) = d(x_1, x_3) = d(x_1, x_4) = d(x_1, x_5)
	\end{align*}
	Any of them is a valid solution to the nearest neighbor problem for $x_1$.\\\\
	The search for a nearest neighbor is a well understood problem \libref{nnsOld, nnsNew} and has many
	applications. Without restrictions, solving the problem on general metrics is proven to
	require $\Omega(n)$ time \libref{nnsOld}, where $n$ is the amount of points.
	
	Typical approaches divide the space into regions, exploiting properties of the metric space.
	Common examples include {\kdTree}s \libref{kdTree}, {\vpTree}s \libref{vpTree},
	{\bkTree}s \libref{bkTree} and {\coverTree}s \libref{coverTree}.\\\\
	The problem also has a lot of variants. We elaborate on two of them:
	\begin{mydef}\label{kNearestNeighborsDef}
		The \textnormal{k-nearest neighbors} of a point $x \in M$ are the
		$k$ closest points $\{y_1, y_2, \ldots, y_k\} \subseteq M$ to $x$. That is
		\begin{align*}
			y_1	&= \argmin_{y' \in M \setminus \{x\}} d(x, y'),\\
			y_2	&= \argmin_{y' \in M \setminus \{x, y_1\}} d(x, y'),\\
				&\vdots\\
			y_k	&= \argmin_{y' \in M \setminus \{x, y_1, \ldots, y_{k - 1}\}} d(x, y').
		\end{align*}
	\end{mydef}
	\begin{mydef}
		The \textnormal{k-neighborhood} of a point $x \in M$ is the set
		\begin{align*}
			\{y \in M \setminus \{x\} | d(x, y) \le k\}.
		\end{align*}
	\end{mydef}

%Cover tree
\section{Cover tree}
	\begin{mydef}\label{coverTree}
		A \textnormal{cover tree} $T$ on a metric space $(M, d)$ is a leveled tree $(V, E)$.
		
		The root is placed at the greatest level, denoted by $i_{\vmax} \in \mathbb{Z}$.
		The level of a node $v \in V$ is
		\begin{align*}
			\lvl(v)	&= i_{\vmax} - \depth(v).
		\end{align*}
		The lowest level is denoted by $i_{\vmin}$.
		Every node $v \in V$ is associated with a point $m \in M$. We write $\assoc(v) = m$.
		Nodes of a certain level form a \textit{cover} of points in $M$. A cover for a level $i$ is defined as
		\begin{align*}
			C_i	&= \{m \in M | \exists v \in V : \lvl(v) = i \land \assoc(v) = m\}.
		\end{align*}
		
		The following properties must hold
		\begin{itemize}
			\item[1.] For a level $i$ there must not exist nodes which are associated with the same point $m \in M$:
				\begin{align*}
					\nexists v, v' \in V : i = \lvl(v) = \lvl(v') \land v \neq v' \land \assoc(v) = \assoc(v')
				\end{align*}
				So each point can at most appear once per level.
			\item[2.] $C_i \subset C_{i - 1}$. This ensures that, once a point was associated with a node in a
				level, it appears in all lower levels too.
			\item[3.] Points are covered by their parents:
				\begin{align*}
					\forall p \in C_{i - 1} \exists q \in C_{i}: d(p, q) < 2^i
				\end{align*}
				and the node $v_p$ with $\lvl(v_p) = i \land \assoc(v_p) = p$ is the parent of the node
				$v_q$ with $\lvl(v_q) = i - 1 \land \assoc(v_q) = q$.
			\item[4.] Points in a cover $C_i$ have a separation of at least $2^i$, i.e.
				\begin{align*}
					\forall p, q \in C_i : p \neq q \Rightarrow d(p, q) > 2^i.
				\end{align*}
		\end{itemize}
	\end{mydef}
	A cover tree \libref{coverTree} has interesting distance properties on its nodes which allows for
	efficient retrieval of nearest neighbors. The general approach is straightforward. Given a node $v$ in the
	tree placed at level $i$, we know that all nodes of the subtree rooted at $v$ are associated with points
	inside a distance of at most $2^i$. This means that, if we search for a nearest neighbor,
	and traverse to a node $v$ in the tree, all nodes underneath $v$ are relatively close to $v$. So, if we already have a
	candidate for a nearest neighbor, with a distance of $d$ and $v$ is already further away than $d + 2^i$;
	$v$ and all nodes in its subtree can not improve the distance.
	% Cover tree example
	\begin{figure}[!ht]
		 \begin{center}
			\begin{tikzpicture}[y = -1cm]
				% Levels
			 	\draw[redArea] (-0.6, -0.6) rectangle (10.6, 0.6);
			 	\node[left] at (-1, 0) {level $6$};
			 	
			 	\draw[blueArea] (-0.6, 1.4) rectangle (10.6, 2.6);
			 	\node[left] at (-1, 2) {level $5$};
			 	
			 	\draw[greenArea] (-0.6, 3.4) rectangle (10.6, 4.6);
			 	\node[left] at (-1, 4) {level $4$};
			 	
			 	\draw[orangeArea] (-0.6, 5.4) rectangle (10.6, 6.6);
			 	\node[left] at (-1, 6) {level $3$};
			 	
			 	% Nodes
			 	% Level 6
			 	\node[minimum size=9mm, circle, draw, fill=red!30] (6-1) at (3, 0) {$x_1$};
			 				 	
			 	% Level 5
			 	\node[minimum size=9mm, circle, draw, fill=blue!30] (5-11) at (0, 2) {$x_{11}$};
			 	\node[minimum size=9mm, circle, draw, fill=blue!30] (5-1) at (5.5, 2) {$x_1$};
			 	
			 	% Level 4
			 	\node[minimum size=9mm, circle, draw, fill=darkgreen!30] (4-11) at (0, 4) {$x_{11}$};
			 	\node[minimum size=9mm, circle, draw, fill=darkgreen!30] (4-1) at (1, 4) {$x_1$};
			 	\node[minimum size=9mm, circle, draw, fill=darkgreen!30] (4-2) at (3, 4) {$x_2$};
			 	\node[minimum size=9mm, circle, draw, fill=darkgreen!30] (4-3) at (5.5, 4) {$x_3$};
			 	\node[minimum size=9mm, circle, draw, fill=darkgreen!30] (4-4) at (8, 4) {$x_4$};
			 	\node[minimum size=9mm, circle, draw, fill=darkgreen!30] (4-5) at (10, 4) {$x_5$};
			 	
			 	% Level 3
			 	\node[minimum size=9mm, circle, draw, fill=orange!30] (3-11) at (0, 6) {$x_{11}$};
			 	\node[minimum size=9mm, circle, draw, fill=orange!30] (3-1) at (1, 6) {$x_1$};
			 	\node[minimum size=9mm, circle, draw, fill=orange!30] (3-2) at (2, 6) {$x_2$};
			 	\node[minimum size=9mm, circle, draw, fill=orange!30] (3-6) at (3, 6) {$x_6$};
			 	\node[minimum size=9mm, circle, draw, fill=orange!30] (3-7) at (4, 6) {$x_7$};
			 	\node[minimum size=9mm, circle, draw, fill=orange!30] (3-3) at (5, 6) {$x_3$};
			 	\node[minimum size=9mm, circle, draw, fill=orange!30] (3-10) at (6, 6) {$x_{10}$};
			 	\node[minimum size=9mm, circle, draw, fill=orange!30] (3-4) at (7, 6) {$x_4$};
			 	\node[minimum size=9mm, circle, draw, fill=orange!30] (3-8) at (8, 6) {$x_8$};
			 	\node[minimum size=9mm, circle, draw, fill=orange!30] (3-9) at (9, 6) {$x_9$};
			 	\node[minimum size=9mm, circle, draw, fill=orange!30] (3-5) at (10, 6) {$x_5$};
			 	
			 	% Edges
			 	% Level 6 to Level 5
			 	\draw[thick, ->] (6-1) to (5-11);
			 	\draw[thick, ->] (6-1) to (5-1);
			 	
			 	% Level 5 to Level 4
			 	\draw[thick, ->] (5-11) to (4-11);
			 	\draw[thick, ->] (5-1) to (4-1);
			 	\draw[thick, ->] (5-1) to (4-2);
			 	\draw[thick, ->] (5-1) to (4-3);
			 	\draw[thick, ->] (5-1) to (4-4);
			 	\draw[thick, ->] (5-1) to (4-5);
			 	
			 	% Level 4 to Level 3
			 	\draw[thick, ->] (4-11) to (3-11);
			 	\draw[thick, ->] (4-1) to (3-1);
			 	\draw[thick, ->] (4-2) to (3-2);
			 	\draw[thick, ->] (4-2) to (3-6);
			 	\draw[thick, ->] (4-2) to (3-7);
			 	\draw[thick, ->] (4-3) to (3-3);
			 	\draw[thick, ->] (4-3) to (3-10);
			 	\draw[thick, ->] (4-4) to (3-4);
			 	\draw[thick, ->] (4-4) to (3-8);
			 	\draw[thick, ->] (4-4) to (3-9);
			 	\draw[thick, ->] (4-5) to (3-5);
			\end{tikzpicture}
		\end{center}
		\caption{Cover tree for the data set of \figref{nearestNeighborProblemExample}.
			Nodes are vertically grouped by their levels and highlighted accordingly.}
		\label{coverTreeExample}
	\end{figure}
	% Cover tree example range
	\begin{figure}[!ht]
		 \begin{center}
		 	% Level 6
			\begin{tikzpicture}[scale=0.6]
				% Grid
				\foreach \i [evaluate=\i as \ii using int(\i*10)] in {0,...,9} {
					\draw [very thin,gray] (\i,0) -- (\i,9);
					\draw [very thin,gray] (0,\i) -- (9,\i);
				}
				
			 	% Nodes
			 	\node[inner sep=2pt, fill=red, circle, draw] (x1) at (5,5) {};
			 	\node[below right] at (x1) {$x_1$};
			 	
			 	\node[inner sep=2pt, fill=black, circle, draw] (x2) at (3,3) {};
			 	\node[below right] at (x2) {$x_2$};
			 	
			 	\node[inner sep=2pt, fill=black, circle, draw] (x3) at (3,7) {};
			 	\node[below right] at (x3) {$x_3$};
			 	
			 	\node[inner sep=2pt, fill=black, circle, draw] (x4) at (7,3) {};
			 	\node[below right] at (x4) {$x_4$};
			 	
			 	\node[inner sep=2pt, fill=black, circle, draw] (x5) at (7,7) {};
			 	\node[below right] at (x5) {$x_5$};
			 	
			 	\node[inner sep=2pt, fill=black, circle, draw] (x6) at (3,1.5) {};
			 	\node[below right] at (x6) {$x_6$};
			 	
			 	\node[inner sep=2pt, fill=black, circle, draw] (x7) at (2,3) {};
			 	\node[below right] at (x7) {$x_7$};
			 	
			 	\node[inner sep=2pt, fill=black, circle, draw] (x8) at (7,1.5) {};
			 	\node[below right] at (x8) {$x_8$};
			 	
			 	\node[inner sep=2pt, fill=black, circle, draw] (x9) at (8.5,3) {};
			 	\node[below right] at (x9) {$x_9$};
			 	
			 	\node[inner sep=2pt, fill=black, circle, draw] (x10) at (2,7) {};
			 	\node[below right] at (x10) {$x_{10}$};
			 	
			 	\node[inner sep=2pt, fill=black, circle, draw] (x11) at (1,8) {};
			 	\node[below right] at (x11) {$x_{11}$};
			 	
			 	% Circles
			 	\draw[clip] (0, 0) rectangle (9, 9);
			 	\node[inner sep=77pt, circle, draw=red, opacity=0.5, pattern=north west lines, pattern color=red] at (x1) {};
			\end{tikzpicture}
			% Level 5
			\begin{tikzpicture}[scale=0.6]
				% Grid
				\foreach \i [evaluate=\i as \ii using int(\i*10)] in {0,...,9} {
					\draw [very thin,gray] (\i,0) -- (\i,9);
					\draw [very thin,gray] (0,\i) -- (9,\i);
				}
				
			 	% Nodes
			 	\node[inner sep=2pt, fill=blue, circle, draw] (x1) at (5,5) {};
			 	\node[below right] at (x1) {$x_1$};
			 	
			 	\node[inner sep=2pt, fill=black, circle, draw] (x2) at (3,3) {};
			 	\node[below right] at (x2) {$x_2$};
			 	
			 	\node[inner sep=2pt, fill=black, circle, draw] (x3) at (3,7) {};
			 	\node[below right] at (x3) {$x_3$};
			 	
			 	\node[inner sep=2pt, fill=black, circle, draw] (x4) at (7,3) {};
			 	\node[below right] at (x4) {$x_4$};
			 	
			 	\node[inner sep=2pt, fill=black, circle, draw] (x5) at (7,7) {};
			 	\node[below right] at (x5) {$x_5$};
			 	
			 	\node[inner sep=2pt, fill=black, circle, draw] (x6) at (3,1.5) {};
			 	\node[below right] at (x6) {$x_6$};
			 	
			 	\node[inner sep=2pt, fill=black, circle, draw] (x7) at (2,3) {};
			 	\node[below right] at (x7) {$x_7$};
			 	
			 	\node[inner sep=2pt, fill=black, circle, draw] (x8) at (7,1.5) {};
			 	\node[below right] at (x8) {$x_8$};
			 	
			 	\node[inner sep=2pt, fill=black, circle, draw] (x9) at (8.5,3) {};
			 	\node[below right] at (x9) {$x_9$};
			 	
			 	\node[inner sep=2pt, fill=black, circle, draw] (x10) at (2,7) {};
			 	\node[below right] at (x10) {$x_{10}$};
			 	
			 	\node[inner sep=2pt, fill=blue, circle, draw] (x11) at (1,8) {};
			 	\node[below right] at (x11) {$x_{11}$};
			 	
			 	% Circles
			 	\draw[clip] (0, 0) rectangle (9, 9);
			 	\node[inner sep=38pt, circle, draw=blue, opacity=0.5, pattern=dots, pattern color=blue] at (x1) {};
			 	\node[inner sep=38pt, circle, draw=blue, opacity=0.5, pattern=dots, pattern color=blue] at (x11) {};
			\end{tikzpicture}\\\phantom{v}\quad\\
			% Level 4
			\begin{tikzpicture}[scale=0.6]
				% Grid
				\foreach \i [evaluate=\i as \ii using int(\i*10)] in {0,...,9} {
					\draw [very thin,gray] (\i,0) -- (\i,9);
					\draw [very thin,gray] (0,\i) -- (9,\i);
				}
				
			 	% Nodes
			 	\node[inner sep=2pt, fill=darkgreen, circle, draw] (x1) at (5,5) {};
			 	\node[below right] at (x1) {$x_1$};
			 	
			 	\node[inner sep=2pt, fill=darkgreen, circle, draw] (x2) at (3,3) {};
			 	\node[below right] at (x2) {$x_2$};
			 	
			 	\node[inner sep=2pt, fill=darkgreen, circle, draw] (x3) at (3,7) {};
			 	\node[below right] at (x3) {$x_3$};
			 	
			 	\node[inner sep=2pt, fill=darkgreen, circle, draw] (x4) at (7,3) {};
			 	\node[below right] at (x4) {$x_4$};
			 	
			 	\node[inner sep=2pt, fill=darkgreen, circle, draw] (x5) at (7,7) {};
			 	\node[below right] at (x5) {$x_5$};
			 	
			 	\node[inner sep=2pt, fill=black, circle, draw] (x6) at (3,1.5) {};
			 	\node[below right] at (x6) {$x_6$};
			 	
			 	\node[inner sep=2pt, fill=black, circle, draw] (x7) at (2,3) {};
			 	\node[below right] at (x7) {$x_7$};
			 	
			 	\node[inner sep=2pt, fill=black, circle, draw] (x8) at (7,1.5) {};
			 	\node[below right] at (x8) {$x_8$};
			 	
			 	\node[inner sep=2pt, fill=black, circle, draw] (x9) at (8.5,3) {};
			 	\node[below right] at (x9) {$x_9$};
			 	
			 	\node[inner sep=2pt, fill=black, circle, draw] (x10) at (2,7) {};
			 	\node[below right] at (x10) {$x_{10}$};
			 	
			 	\node[inner sep=2pt, fill=darkgreen, circle, draw] (x11) at (1,8) {};
			 	\node[below right] at (x11) {$x_{11}$};
			 	
			 	% Circles
			 	\draw[clip] (0, 0) rectangle (9, 9);
			 	\node[inner sep=19pt, circle, draw=darkgreen, opacity=0.5, pattern=horizontal lines, pattern color=darkgreen] at (x1) {};
			 	\node[inner sep=19pt, circle, draw=darkgreen, opacity=0.5, pattern=horizontal lines, pattern color=darkgreen] at (x2) {};
			 	\node[inner sep=19pt, circle, draw=darkgreen, opacity=0.5, pattern=horizontal lines, pattern color=darkgreen] at (x3) {};
			 	\node[inner sep=19pt, circle, draw=darkgreen, opacity=0.5, pattern=horizontal lines, pattern color=darkgreen] at (x4) {};
			 	\node[inner sep=19pt, circle, draw=darkgreen, opacity=0.5, pattern=horizontal lines, pattern color=darkgreen] at (x5) {};
			 	\node[inner sep=19pt, circle, draw=darkgreen, opacity=0.5, pattern=horizontal lines, pattern color=darkgreen] at (x11) {};
			\end{tikzpicture}
			% Level 3
			\begin{tikzpicture}[scale=0.6]
				% Grid
				\foreach \i [evaluate=\i as \ii using int(\i*10)] in {0,...,9} {
					\draw [very thin,gray] (\i,0) -- (\i,9);
					\draw [very thin,gray] (0,\i) -- (9,\i);
				}
				
			 	% Nodes
			 	\node[inner sep=2pt, fill=orange, circle, draw] (x1) at (5,5) {};
			 	\node[below right] at (x1) {$x_1$};
			 	
			 	\node[inner sep=2pt, fill=orange, circle, draw] (x2) at (3,3) {};
			 	\node[below right] at (x2) {$x_2$};
			 	
			 	\node[inner sep=2pt, fill=orange, circle, draw] (x3) at (3,7) {};
			 	\node[below right] at (x3) {$x_3$};
			 	
			 	\node[inner sep=2pt, fill=orange, circle, draw] (x4) at (7,3) {};
			 	\node[below right] at (x4) {$x_4$};
			 	
			 	\node[inner sep=2pt, fill=orange, circle, draw] (x5) at (7,7) {};
			 	\node[below right] at (x5) {$x_5$};
			 	
			 	\node[inner sep=2pt, fill=orange, circle, draw] (x6) at (3,1.5) {};
			 	\node[below right] at (x6) {$x_6$};
			 	
			 	\node[inner sep=2pt, fill=orange, circle, draw] (x7) at (2,3) {};
			 	\node[below right] at (x7) {$x_7$};
			 	
			 	\node[inner sep=2pt, fill=orange, circle, draw] (x8) at (7,1.5) {};
			 	\node[below right] at (x8) {$x_8$};
			 	
			 	\node[inner sep=2pt, fill=orange, circle, draw] (x9) at (8.5,3) {};
			 	\node[below right] at (x9) {$x_9$};
			 	
			 	\node[inner sep=2pt, fill=orange, circle, draw] (x10) at (2,7) {};
			 	\node[below right] at (x10) {$x_{10}$};
			 	
			 	\node[inner sep=2pt, fill=orange, circle, draw] (x11) at (1,8) {};
			 	\node[below right] at (x11) {$x_{11}$};
			 	
			 	% Circles
			 	\draw[clip] (0, 0) rectangle (9, 9);
			 	\node[inner sep=10pt, circle, draw=orange, opacity=0.5, pattern=crosshatch, pattern color=orange] at (x1) {};
			 	\node[inner sep=10pt, circle, draw=orange, opacity=0.5, pattern=crosshatch, pattern color=orange] at (x2) {};
			 	\node[inner sep=10pt, circle, draw=orange, opacity=0.5, pattern=crosshatch, pattern color=orange] at (x3) {};
			 	\node[inner sep=10pt, circle, draw=orange, opacity=0.5, pattern=crosshatch, pattern color=orange] at (x4) {};
			 	\node[inner sep=10pt, circle, draw=orange, opacity=0.5, pattern=crosshatch, pattern color=orange] at (x5) {};
			 	\node[inner sep= 10pt, circle, draw=orange, opacity=0.5, pattern=crosshatch, pattern color=orange] at (x6) {};
			 	\node[inner sep=10pt, circle, draw=orange, opacity=0.5, pattern=crosshatch, pattern color=orange] at (x7) {};
			 	\node[inner sep=10pt, circle, draw=orange, opacity=0.5, pattern=crosshatch, pattern color=orange] at (x8) {};
			 	\node[inner sep=10pt, circle, draw=orange, opacity=0.5, pattern=crosshatch, pattern color=orange] at (x9) {};
			 	\node[inner sep=10pt, circle, draw=orange, opacity=0.5, pattern=crosshatch, pattern color=orange] at (x10) {};
			 	\node[inner sep=10pt, circle, draw=orange, opacity=0.5, pattern=crosshatch, pattern color=orange] at (x11) {};
			\end{tikzpicture}
		\end{center}
		\caption{Figure that shows the separation property for each level of the cover tree shown by \figref{coverTreeExample}.
			The levels are highlighted in the same manner than in the previous example. The levels are $6, 5, 4$ and $3$ from
			top left to bottom right. The radii around the points have a size of $2^6, 2^5, 2^4$ and $2^3$.}
		\label{coverTreeExampleRange}
	\end{figure}\quad\\
	\figref{coverTreeExample} shows a valid cover tree for the toy example illustrated by \figref{nearestNeighborProblemExample}.
	The covers are
	\begin{align*}
		C_6	&= \{x_1\},\\
		C_5	&= \{x_1, x_{11}\},\\
		C_4	&= \{x_1, x_2, x_3, x_4, x_5, x_{11}\},\\
		C_3	&= \{x_1, x_2, x_3, x_4, x_5, x_6, x_7, x_8, x_9, x_{10}, x_{11}\}.
	\end{align*}
	Clearly the first property holds, there is no level where a $x_i$ is associated with a node more than once.
	The second property holds too, it is
	\begin{align*}
		C_6 \subset C_5 \subset C_4 \subset C_3.
	\end{align*}
	For the last two properties we take a look at \figref{coverTreeExampleRange}. It illustrates the fourth property.
	The property states that all points in a cover $C_i$ must have a distance of at least $2^i$ to each other.
	For level $6$ this is trivial since the set only contains $x_1$. For level $5$ it must hold that
	\begin{align*}
		d(x_1, x_{11}) = 50 > 32 = 2^5,
	\end{align*}
	which is true. If this would not be the case, the figure would show the nodes included inside the circle
	around the other node. Analogously all nodes in $C_4$ and $C_3$ are separated enough from each other.\\\\
	The third property can easily be confirmed using the figure too. It states that a node in level $i - 1$ must
	be closer than $2^i$ to its parent. Obviously this holds for $x_1$ and $x_{11}$ in level $5$, as a radius
	of $2^6$ around their parent $x_1$ covers all nodes. Likewise are $x_1, x_2, x_3, x_4$ and $x_5$ included
	in the circle around their parent $x_1$ with radius $2^5$.
	
	Note that it is not necessary that a node covers its whole subtree in its level. As example, we refer to $x_1$ in level $5$
	which does not cover $x_{10}$, as $d(x_1, x_{10}) > 2^5$, though it is part of the subtree rooted at $x_1$. The third property only demands
	that a parent covers all its direct children, not grandchildren or similar.\\\\
	% Cover tree insertion
	\IncMargin{1em}
	\begin{algorithm}
		\SetKwInOut{Input}{input}
  		\SetKwInOut{Output}{output}
		\SetKwFunction{d}{d}\SetKwFunction{fchildren}{children}\SetKwFunction{insert}{insert}
		\BlankLine
		\Input{point $p \in M$, candidate cover set $Q_i \subseteq C_i$, level $i$}
		\Output{\true if $p$ was inserted at level $i - 1$, \false otherwise}
		\BlankLine
		$Q \leftarrow \{\fchildren(q) | q \in Q_i\}$\;
		\BlankLine
		\If{$d(p, Q) > 2^i$}{
			\Return \false\tcp*{Check separation}
		}\Else{
			$Q_{i - 1} \leftarrow \{q \in Q | \d(p, q) \le 2^i\}$\tcp*{Covering candidates}
			\BlankLine
			\If{$\neg\insert(p, Q_{i - 1}, i - 1) \land \d(p, Q_i) \le 2^i$}{
				pick any $q \in Q_{i} : \d(p, q) \le 2^i$\;
				append $q$ as child to $q$\;
				\Return \true\;
			}\Else{
				\Return \false\;
			}
		}
		\BlankLine
		\caption{Inserting a point into a cover tree operating on a metric space $(M, d)$.}\label{coverTreeInsert}
	\end{algorithm}\DecMargin{1em}\quad\\
	The cover tree is constructed using \algoref{coverTreeInsert} with the maximal level $i_{\vmax}$ and the cover
	set $C_k$ which only consists of the root. The algorithm is stated recursively, but can easily be implemented
	without recursion by descending the levels and only following relevant candidates.
	
	A point $p$ can be appended in level $i - 1$ to a parent $q$ in level $i$ if the point has enough separation to all other
	nodes in this level, meaning more than $2^{i - 1}$, and is covered by the parent, that is a distance of less than $2^i$.
	The algorithm searches such a point by descending the levels, computing the separation and appending it to a
	node if it also covers the point.\\\\
	% Cover tree nearest neighbor search
	\IncMargin{1em}
	\begin{algorithm}
		\SetKwInOut{Input}{input}
  		\SetKwInOut{Output}{output}
		\SetKwFunction{d}{d}\SetKwFunction{fchildren}{children}
		\BlankLine
		\Input{point $p \in M$}
		\Output{a nearest neighbor to $p$ in $M$}
		\BlankLine
		$Q_{i_{\vmax}} \leftarrow C_{i_{\vmax}}$\;
		\For{$i$ from $i_{\vmax}$ to $i_{\vmin}$}{
			$Q \leftarrow \{\fchildren(q) | q \in Q_i\}$\;
			$Q_{i - 1} \leftarrow \{q \in Q | \d(p, q) \le \d(p, Q) + 2^i\}$\;
		}
		$\Return \argmin_{q \in Q_{i_{\vmin}}} \d(p, q)$\;
		\BlankLine
		\caption{Searching a nearest neighbor in a cover tree operating on a metric space $(M, d)$.}\label{coverTreeSearch}
	\end{algorithm}\DecMargin{1em}\quad\\
	A search for a nearest neighbor follows a similar approach. \algoref{coverTreeSearch} starts at the root and traverses
	the tree by following the children. The candidate set is refined by only following children which are closer than
	\begin{align*}
		d(p, Q) + 2^i.
	\end{align*}
	There, the distance to the set represents the distance of the currently best candidate. Nodes in the subtree
	rooted at a child can maximally be $2^i$ closer than the child itself. Therefore, take a look
	at \figref{coverTreeExampleRange} where $x_2$ is maximally $2^5$ closer to $x_7$ than $x_1$, else it would not
	be covered by its parent $x_1$.
	Because of that the algorithm only follows children which can have nodes in their subtree
	that improve over the currently best candidate. Other children are rejected.
	
	Note that the algorithm must track down all levels, as another node could show up in the lowest level because of
	the separation property.\\\\
	% Cover tree k-nearest neighbor search
	\IncMargin{1em}
	\begin{algorithm}
		\SetKwInOut{Input}{input}
  		\SetKwInOut{Output}{output}
		\SetKwFunction{d}{d}\SetKwFunction{fchildren}{children}
		\BlankLine
		\Input{point $p \in M$, amount $k \in \mathbb{N}$}
		\Output{$k$-nearest neighbors to $p$ in $M$}
		\BlankLine
		$Q_{i_{\vmax}} \leftarrow C_{i_{\vmax}}$\;
		\For{$i$ from $i_{\vmax}$ to $i_{\vmin}$}{
			$Q \leftarrow \{\fchildren(q) | q \in Q_i\}$\;
			\BlankLine
			perform a $k$-partial sort of $Q$, ascending in $\d(p, q)$\;
			let $q'$ be the $k$-th element of $Q$\;
			\BlankLine
			$Q_{i - 1} \leftarrow \{q \in Q | \d(p, q) \le \d(p, q') + 2^i\}$\;
		}
		\BlankLine
		perform a $k$-partial sort of $Q_{i_{\vmin}}$, ascending in $\d(p, q)$\;
		\Return first $k$ elements of $Q_{i_{\vmin}}$\;
		\BlankLine
		\caption{Searching the $k$-nearest neighbors in a cover tree operating on a metric space $(M, d)$.}\label{coverTreeKSearch}
	\end{algorithm}\DecMargin{1em}\quad\\
	% Cover tree k-neighborhood computation
	\IncMargin{1em}
	\begin{algorithm}
		\SetKwInOut{Input}{input}
  		\SetKwInOut{Output}{output}
		\SetKwFunction{d}{d}\SetKwFunction{fchildren}{children}
		\BlankLine
		\Input{point $p \in M$, radius $k \in \mathbb{R}_{\ge 0}$}
		\Output{$k$-neighborhood of $p$ in $M$}
		\BlankLine
		$Q_{i_{\vmax}} \leftarrow C_{i_{\vmax}}$\;
		\For{$i$ from $i_{\vmax}$ to $i_{\vmin}$}{
			$Q \leftarrow \{\fchildren(q) | q \in Q_i\}$\;
			$Q_{i - 1} \leftarrow \{q \in Q | \d(p, q) \le k + 2^i\}$\;
		}
		$\Return \{q \in Q_{i_{\vmin}} | \d(p, q) \le k\}$\;
		\BlankLine
		\caption{Computing the $k$-neighborhood by using a cover tree which operates on a metric space $(M, d)$.}\label{coverTreeKNeighborhood}
	\end{algorithm}\DecMargin{1em}\quad\\
	The cover tree can also be used to efficiently compute the $k$-nearest neighbors or the $k$-neighborhood.
	In order to compute the $k$-nearest neighbors, \algoref{coverTreeKSearch} extends the range bound from the currently
	best candidate to the $k$-th best candidate. Likewise does \algoref{coverTreeKNeighborhood} extend the bound to
	the given range $k$ instead of involving candidate distances.\\\\
	For other operations and a detailed analysis of the cover tree, as well as its complexity and a comparison
	against other techniques, refer to \libref{coverTree}.
%Shortest path problem
\section{Shortest path problem}\label{shortestPathProblem}
	For route planning, routes through a network must be optimized in regards to one or even many criteria.
	A common criteria is the \textit{travel time}. Others include cost, number of transfers or restrictions
	in transportation types.
	
	In this chapter, we will first give an informal description of the \earliestArrivalProblem. Followed by
	the \shortestPathProblem, which is equivalent to the \earliestArrivalProblem for our graph-based network
	representations.
	
	Then, we introduce algorithms for solving the problem. First, for time-independent networks, then for time-dependent.
	Afterwards, we explain two solutions for combined networks, using multiple transportation modes. There, the problem
	description slightly changes by adding transportation mode restrictions.\\\\
	\begin{mydef}
		The earliest arrival problem asks for finding a \textnormal{route} in a network with following properties
		\begin{itemize}
			\item[1.] The route must start at $s$ and end at $t$.
			\item[2.] The departure time at $s$ is $\tau$.
			\item[3.] All other applicable routes must have a greater travel time, i.e. arrive later at $t$.
		\end{itemize}
		Points $s$ and $t$ are given source and target points in the network respectively. $\tau$ is the desired departure time,
		it may be ignored for a time-independent network.
	\end{mydef}
	\begin{mydef}
		Given a graph $G = (V, E)$, source and target nodes $s, t \in V$ and a desired departure time $\tau$, the shortest path
		problem asks for a path $p$ (see \defref{path}) which
		\begin{itemize}
			\item[1.] begins at $s$ and ends at $t$,
			\item[2.] has the smallest weight of all applicable paths.
		\end{itemize}
		The arrival time at $t$ is $\tau$ plus the weight of $p$. In a time-dependent
		graph $\tau$ must be used to ensure correct edge weights. The path $p$ is called \textnormal{shortest path}.
	\end{mydef}\quad\\
	Additionally, we consider a special variant of the shortest path problem:
	\begin{mydef}
		The many-to-one shortest path problem is a variation of the shortest path problem
		where the source consists of a set of source nodes $S \subseteq V$.

		The problem asks for the path $p$ that starts at the source $s \in S$ which minimizes the path weight.
	\end{mydef}

%Time-independent
\subsection{Time-independent}
	Route planning in time-independent networks is a very well studied problem.
	Many efficient solutions to the shortest path problem exists. We introduce a very basic algorithm, \dijkstra
	and a simple improvement based on heuristics, \astar.
	% Time-independent example
	\begin{figure}[!ht]
		 \begin{center}
			\begin{tikzpicture}[y = -1cm]
			 	% Nodes
			 	\node[circle, draw] (v1) at (0, 0) {$v_1$};
			 	\node[circle, draw] (v2) at (2, 0) {$v_2$};
			 	\node[circle, draw] (v3) at (0, 2) {$v_3$};
			 	\node[circle, draw] (v4) at (2, 2) {$v_4$};
			 	\node[circle, draw] (v5) at (4, 0) {$v_5$};
			 	
			 	% Edges
			 	\draw[ultra thick, ->, color = red] (v1) to node[above] {$8$} (v2);
			 	\draw[ultra thick, ->, color = red] (v2) to node[above] {$2$} (v5);
			 	\draw[->] (v1) to node[left] {$1$} (v3);
			 	\draw[->] (v3) to node[above] {$2$} (v4);
			 	\draw[->] (v4) to node[left] {$1$} (v2);
			 	\draw[->] (v4) to node[below right] {$4$} (v5);
			\end{tikzpicture}\qquad\qquad\qquad
			\begin{tikzpicture}[y = -1cm]
			 	% Nodes
			 	\node[circle, draw] (v1) at (0, 0) {$v_1$};
			 	\node[circle, draw] (v2) at (2, 0) {$v_2$};
			 	\node[circle, draw] (v3) at (0, 2) {$v_3$};
			 	\node[circle, draw] (v4) at (2, 2) {$v_4$};
			 	\node[circle, draw] (v5) at (4, 0) {$v_5$};
			 	
			 	% Edges
			 	\draw[->] (v1) to node[above] {$8$} (v2);
			 	\draw[->] (v2) to node[above] {$2$} (v5);
			 	\draw[ultra thick, ->, color = blue] (v1) to node[left] {$1$} (v3);
			 	\draw[ultra thick, ->, color = blue] (v3) to node[above] {$2$} (v4);
			 	\draw[->] (v4) to node[left] {$1$} (v2);
			 	\draw[ultra thick, ->, color = blue] (v4) to node[below right] {$4$} (v5);
			\end{tikzpicture}\quad\\\phantom{v}\quad\\
			\begin{tikzpicture}[y = -1cm]
			 	% Nodes
			 	\node[circle, draw] (v1) at (0, 0) {$v_1$};
			 	\node[circle, draw] (v2) at (2, 0) {$v_2$};
			 	\node[circle, draw] (v3) at (0, 2) {$v_3$};
			 	\node[circle, draw] (v4) at (2, 2) {$v_4$};
			 	\node[circle, draw] (v5) at (4, 0) {$v_5$};
			 	
			 	% Edges
			 	\draw[->] (v1) to node[above] {$8$} (v2);
			 	\draw[ultra thick, ->, color = darkgreen] (v2) to node[above] {$2$} (v5);
			 	\draw[ultra thick, ->, color = darkgreen] (v1) to node[left] {$1$} (v3);
			 	\draw[ultra thick, ->, color = darkgreen] (v3) to node[above] {$2$} (v4);
			 	\draw[ultra thick, ->, color = darkgreen] (v4) to node[left] {$1$} (v2);
			 	\draw[->] (v4) to node[below right] {$4$} (v5);
			\end{tikzpicture}
		\end{center}
		\caption{Example for a time independent network, represented by a road graph.
		The figure shows three paths from $v_1$ to $v_5$. From top left to bottom right, the path
		weights are $10$, $7$ and $6$. The last example represents the shortest path from $v_1$ to $v_5$.}
		\label{timeIndependentExample}
	\end{figure}\quad\\
	The network shown by \figref{timeIndependentExample} acts as toy example for this section.

%Dijkstra
\subsubsection{Dijkstra}
	\dijkstra \libref{dijkstra} is a simple approach to solving the shortest path problem. It can be viewed
	as the logical extension of breadth-first search (\bfs) \libref{dijkstra} in weighted graphs. The algorithm
	revolves around a priority queue where it stores neighboring nodes, sorted by their shortest path cost.
	In each round, the node with the smallest shortest path cost is \textit{relaxed}. That is, all its neighboring,
	not already relaxed, nodes are added to the queue. The algorithm terminates as soon as the target node has been relaxed.
	\algoref{dijkstra} gives a formal description.
	% Dijkstra implementation
	\IncMargin{1em}
	\begin{algorithm}
		\SetKwInOut{Input}{input}
  		\SetKwInOut{Output}{output}
  		\SetKw{Break}{break}
  		\SetKwData{undef}{undefined}\SetKwData{currentDist}{currentDist}
		\SetKwFunction{dist}{dist}\SetKwFunction{prev}{prev}
		\BlankLine
		\Input{graph $G = (V, E)$, source $s \in V$, target $t \in V$}
		\Output{shortest path from $s$ to $t$}
		\BlankLine
		\tcp{Initialization}
		\For{$v \in V$}{
			$\dist(v) \leftarrow \infty$\;
			$\prev(v) \leftarrow \undef$\;
		}
		\BlankLine
		$\dist(s) \leftarrow 0$\;
		$Q \leftarrow \{s\}$\;
		\BlankLine
		\tcp{Compute shortest paths}
		\While{$Q$ is not empty}{
			$u \leftarrow \argmin_{u' \in Q} \dist(u')$\;
			$Q \leftarrow Q \setminus \{u\}$\;
			\BlankLine
			\If{$u == t$}{
				\Break\;
			}
			\BlankLine
			\tcp{Relax $u$}
			\For{outgoing edge $(u, w, v) \in E$}{
				$\currentDist \leftarrow \dist(u) + w$\;
				\If{$\currentDist < \dist(v)$}{
					\tcp{Improve distance by using this edge}
					$\dist(v) \leftarrow \currentDist$\;
					$\prev(v) \leftarrow u$\;
					$Q \leftarrow Q \cup \{v\}$\;
				}
			}
		}
		\BlankLine
		\tcp{Extract path by backtracking}
		$p \leftarrow$ empty path\;
		$u \leftarrow t$\;
		\While{$\prev(u) \neq \undef$}{
			$w \leftarrow \dist(u) - \dist(\prev(u))$\;
			prepend $(\prev(u), w, u)$ to $p$\;
			$u \leftarrow \prev(u)$\;
		}
		prepend $s$ to $p$\;
		\Return $p$\;
		\BlankLine
		\caption{Dijkstra's algorithm for computing shortest paths in time-independent graphs.}\label{dijkstra}
	\end{algorithm}\DecMargin{1em}\quad\\\\
	To familiarize with the algorithm, we step through the execution for the graph shown by \figref{timeIndependentExample},
	with $v_1$ as source and $v_5$ as target node.
	
	The $\dist$ function, often implemented as array, stores the tentative shortest path weight to the given node.
	$\prev$ is used for path extraction at the end, it stores the parent nodes used for the shortest paths represented by $\dist$.
	The algorithm starts by initializing both collections with default values. Initially the distance to all nodes, except the source, is unknown.
	Thus, $\infty$ is used for them. $Q$ represents the list of nodes that need to be processed, usually implemented as priority queue.
	Initially, it only holds the source node $s$.
	
	In the example $Q$ is initially $\{v_1\}$. The algorithm then relaxes $v_1$ and stores distances to its neighbors:
	\begin{center}
		\begin{tabular}{CC}
			\dist(v_2) = 8	&\prev(v_2) = v_1,\\
			\dist(v_3) = 1	&\prev(v_3) = v_1
		\end{tabular}
	\end{center}
	Additionally, the queue $Q$ is updated, it is
	\begin{align*}
		Q	&= \{v_2, v_3\}.
	\end{align*}
	The next iteration of the loop starts and the node with the smallest distance is chosen, i.e. $v_3$. The node is relaxed and we receive
	\begin{center}
		\begin{tabular}{CC}
			\dist(v_4) = 3	&\prev(v_4) = v_3,\\
			\multicolumn{2}{c}{$Q = \{v_2, v_4\}.$}
		\end{tabular}
	\end{center}
	The next node is $v_4$, yielding
	\begin{center}
		\begin{tabular}{CC}
			\dist(v_2) = 4	&\prev(v_2) = v_4,\\
			\dist(v_5) = 7	&\prev(v_5) = v_4,\\
			\multicolumn{2}{c}{$Q = \{v_2, v_5\}.$}
		\end{tabular}
	\end{center}
	Note that $v_4$ improves the distance to $v_2$. The previous values for $v_2$ are overwritten and the
	tentative shortest path to $v_2$ uses $(v_4, 1, v_2)$ and not $(v_1, 8, v_2)$ anymore.
	In the next round $v_2$ is relaxed which improves the distance to $v_5$:
	\begin{center}
		\begin{tabular}{CC}
			\dist(v_5) = 6	&\prev(v_5) = v_2,\\
			\multicolumn{2}{c}{$Q = \{v_5\}.$}
		\end{tabular}
	\end{center}
	The only node left is the target node $v_5$ now. It is relaxed and the loop terminates.
	The algorithm backtracks the parent pointer
	\begin{align*}
		\prev(v_5)	&= v_2,\\
		\prev(v_2)	&= v_4,\\
		\prev(v_4)	&= v_3,\\
		\prev(v_3)	&= v_1,\\
		\prev(v_1)	&= \vundef
	\end{align*}
	and constructs the shortest path
	\begin{align*}
		p	&= (v_1, 1, v_3)(v_3, 2, v_4)(v_4, 1, v_2)(v_2, 2, v_5)
	\end{align*}
	which is the path shown by the last example in the figure.

%A*
\subsubsection{\astar and \alt}\label{alt}
	An important observation of \dijkstra is that, if it settles the shortest path distance to a node, then,
	all nodes which are closer to the source, were already settled in a previous round.
	
	Moreover, the algorithm explores the graph in all directions equally. It has no sense of \textit{goal direction}.\\\\
	The \astar algorithm \libref{alt} is a simple extension of \dijkstra which improves its efficiency by steering the
	exploration more towards the target. \figref{dijkstra_vs_astar} illustrates this by comparing the \textit{search space}
	of both algorithms. The search space of \astar is smaller and much more directed to the target node $t$.
	% A-star vs Dijkstra
	\begin{figure}[!ht]
		 \begin{center}
			\includegraphics[scale=0.5]{res/dijkstra_vs_astar}
		\end{center}
		\caption{Schematic illustration of a query processed by \dijkstra (left) and \astar (right).
			The highlighted areas indicate the \textit{search space}, i.e. the nodes the algorithm has explored already.
			The illustration is from \libref{routePlanningOverview}.}
		\label{dijkstra_vs_astar}
	\end{figure}\quad\\
	Unfortunately, computing the exact goal direction is equivalent to computing the shortest path to the target.
	Therefore, a heuristic is used to approximate the direction. The choice of the heuristic heavily depends on the underlying network.
	In the worst case, a heuristic may not improve over \dijkstra and the same search space is received. In the best case,
	the algorithm explores only the nodes on the shortest path.
	
	Such a heuristic must fulfill two properties, formulated by \defref{heuristic}.
	\begin{mydef}\label{heuristic}
		Given a graph $G = (V, E)$, a metric $\dist$ on the set $V$ (see \defref{metric}), a \textnormal{heuristic} $h$
		is a metric on $V$ which approximates $\dist$. The heurstic $h$ must be
		\begin{itemize}
			\item[1.] \textnormal{admissable}, i.e. never overestimate:
				\begin{align*}
					\forall u, t \in V: h(u, t) \le \dist(u, t)
				\end{align*}
			\item[2.] \textnormal{monotone}, i.e. satisfy the triangle inequality:
				\begin{align*}
					\forall t \in V\,\forall (u, w, v) \in E: h(u, t) \le w + h(v, t)
				\end{align*}
		\end{itemize}
	\end{mydef}
	\todo{Write something...}

%Time-dependend
\subsection{Time-dependend}
	Blabla

%Connection scan
\subsubsection{Connection scan}\label{csa}
	Blabla

%Multi-modal
\subsection{Multi-modal}
	Blabla

%Modified Dijkstra
\subsubsection{Modified Dijkstra}\label{modifiedDijkstra}
	Blabla

%Access nodes
\subsubsection{Access nodes}\label{accessNodes}
	Blabla

%Other algorithms
\subsection{Other algorithms}
	Blabla
% Evaluation
\section{Evaluation}\label{evaluation}
	Blabla

%Input data
\subsection{Input data}
	Blabla

%Experiments
\subsection{Experiments}
	Blabla

%Nearest neighbor computation
\subsubsection{Nearest neighbor computation}
	Blabla

%Uni-modal routing
\subsubsection{Uni-modal routing}
	Blabla

%Multi-modal routing
\subsubsection{Multi-modal routing}
	Blabla

%Summary
\subsection{Summary}
	Blabla
% Conclusion
\chapter{Conclusion}\label{conclusion}
	Route planning is a problem of much interest that gained a lot of interest in the last decades. Problem settings like \uniModal route planning are
	well researched, efficient solutions were developed. Corresponding research is now focused on \multiModal routing and other difficult problems
	occurring in practice, such as turn penalties and multi criteria routing.

% Future Work
\section{Future Work}
	Our goals for the future are focused on further extending and improving \cobweb. The most important step in order to make our \anr version
	viable is to implement a sophisticated routing algorithm for road networks. Such as techniques based on \textit{contraction}, like contraction
	hierarchies (\ch) \libref{ch} and transit node routing (\tnr) \libref{tnr, tnrReconsidered}. Combined with \csa this should yield promising
	acceptable low query times for shortest path computations.
	
	To improve the quality of our shortest paths, access node selection needs to be improved. It should not solely be based on vicinity.
	Stops should be ordered in a certain priority, measuring their importance for the network. Ideally, a stop is important if it is part
	of many shortest paths. A simple hierarchy can be obtained by counting the amount of connections available at a certain stop.
	The more connections, the more likely it is important.
	The hierarchy can further be fine tuned by injecting query logs of other applications or manually selecting big main
	stations before smaller stops.\\\\
	Another important aspect is to greatly expand the amount of metadata displayed next to a computed journey in the front end.
	An application that is to be used by clients must give extensive information on routes. Not only the name of a street and identification numbers of
	trains, but also including precise information on a road type, possible restrictions, access to the complete schedule of the trip of a transit vehicle,
	cost, and possibly even include forecasts for traffic congestion.
	
	Currently, \cobweb uses a database to store metadata which are not directly relevant to routing. The data is then later, after computing
	the shortest route, retrieved to \textit{annotate} the journey. For efficient retrieval, in particular if the amount of stored metadata increases,
	the database structure needs to be improved. Also, parsing in a new data-feed and inserting missing information into the database does take
	too long at the moment and should be improved.\\\\
	Long term goals consist of adding multi-criteria routing \libref{multiCriteria}, such as optimizing not only for the earliest arrival time but
	also for factors like cost and amount of transfers. And adding support for real-time data (\rtd) \libref{rtd}, for example incorporating traffic
	congestion, road outage and transit vehicle delays.
	Real-time data is already available for most networks, especially for transit networks. However, \rtd is particularly hard to implement,
	because the underlying network changes, possibly invalidating precomputations. Fortunately, only small sections of a
	network are affected and need to be adjusted, leading to the identification of a changes impact and possible recomputations.

% Summary
\section{Summary}
	We have presented common and established models for road and transit networks. Graph based solutions are straightforward representations of the network,
	but can not easily adapt to time dependent data, such as transit networks. Timetables are non-graph based alternatives for public transit networks
	which fit their structure better than static graphs. Additionally, a link graph can be used to combine graph based models for multiple networks in a
	straightforward manner. While it might not necessarily be an effective approach, it makes route planning on combined networks for graph based
	algorithms possible.\\\\
	In order to explain more sophisticated route planning approaches, we presented the \nearestNeighborProblem and thoroughly discussed an
	efficient solution to the problem and various variants, using {\coverTree}s.\\\\
	We covered basic route planning algorithms, such as \dijkstra and common optimizations like \astar. The effectiveness of \astar heavily relies on the chosen
	heuristic, which depends on the underlying structure of the network. \alt was presented as a solution to this problem, providing a general applicable heuristic
	which is based on the actual shortest path distances to chosen landmarks. For an overview of more sophisticated \uniModal time-independent
	algorithms, we refer to \libref{routePlanningOverview}.
	
	\csa was introduced as an efficient approach for time-dependent route planning on timetables. The approach is very simple, it just processes all connections
	available after the initial departure time. \csa is fast because it heavily exploits cache locality \libref{cacheLocality} and other low-level optimizations
	for arrays.
	
	For \multiModal route planning we showed how \dijkstra can be adapted to run on a link graph, representing a combined network. Further, we presented the
	general concept of \anr and proposed a simplified variant of it, generalized to an arbitrary algorithm for road networks and another algorithm for transit
	networks. This makes it possible to combine a graph based solution like \dijkstra, or even more sophisticated approaches, for the road network,
	with a timetable based approach for transit networks, such as \csa.\\\\
	Further, we presented experimental results of implementations in the \cobweb project \libref{cobweb} and discussed them.
	For the experiments three data sets are used, \freiburgR, \stuttgartR and \switzerlandR.
	The setup, as well as the structure of the input data, was thoroughly explained. {\coverTree}s and \csa turned out to be a a very
	efficient solution to their corresponding problems. \dijkstra works well for short range queries,
	but scales bad for increasing ranges. Further, it lacks behind more sophisticated approaches as seen in \figref{uniModalTimeIndependentResultsExternalOverview}.
	\astar using $\asTheCrowFlies$ does not perform well on networks used for road planning. While \alt, if carefully implemented, typically beats \dijkstra,
	especially for mid to long range queries. In practice, link graphs are often not feasible due to extreme demands on space capacity. For \multiModal routing
	\dijkstra performs similar to \uniModal routing, being feasible for short range queries, but scaling bad for increasing ranges. \anr, if paired with efficient algorithms
	for both networks, is a promising approach to \multiModal route planning, as seen in \libref{accessNodeRouting}.\\\\
	Route planning, in particular in practice, is a complex topic. A typical application needs to account for more than just finding a route with the shortest travel time.
	Turn penalties and multi-criteria routing, such as the cost of a trip, are important factors for a client and need to be considered. A similar observation is done
	for \multiModal routing, where transportation mode restrictions, in practice, are not just a set of available modes, but rather a complex model with
	multiple states depending on previous states, as explained in \sectionref{multiModal_sec}.
	
	Most algorithms do not adapt well to such restrictions, leading to the development of many very specialized solutions. Because of that, existing
	approaches, such as \anr, rather try to combine multiple algorithms, all suited well for their own specialized type of network. In particular
	for \multiModal routing, including common restrictions occurring in practice, there does not yet exist a feasible solution for networks of a large scale,
	such as big countries or even continents. However, with increasing research in the last decade, many promising approaches were developed and
	a solution does not seem too far.

\bibliography{literature}
\end{document}