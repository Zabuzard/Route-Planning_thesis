% Evaluation
\section{Evaluation}\label{evaluation}
	In this section we report on our experimental results for the presented algorithms on three data sets of increasing size.
	Therefore, we first give insights on the data sets and how the network models are obtained. Afterwards we evaluate
	{\coverTree}s, \dijkstra, \astar (with $\asTheCrowFlies$), \alt, \csa and \multiModal methods such as the adopted \dijkstra
	and our simplified version of \anr on the given data sets.\\\\
	When evaluating shortest path queries on randomly chosen source and target nodes, the resulting paths tend to be long-range.
	However, in practice most queries are only local and algorithms like \dijkstra do not scale well with increasing range.
	To overcome this measurement problem, we introduce the notion of a \textit{Dijkstra rank} \libref{dijkstraRank}.
	\begin{mydef}\label{dijkstraRank}
		Given a graph $G = (V, E)$, the \textnormal{Dijkstra rank} of a node $v \in V$ is the number of the iteration in which,
		when running \dijkstra on the graph, it is polled from the priority queue (see \textbf{line 7} of \algoref{dijkstra}).
		
		That is the position $i$ for $v_i$ in the order of vertices when sorted ascending by their distance to the source, i.e.
		\begin{align*}
			v_1, v_2, \ldots, v_{|V|}
		\end{align*}
		with $\dist(v_i) \le \dist(v_{i + 1})$ for all $i$.
	\end{mydef}\quad\\
	Instead of choosing queries randomly, we only choose source nodes randomly and then select targets by their
	\textit{Dijkstra rank} to the source. Queries can then be sorted by the \textit{Dijkstra rank} and, by that,
	evaluated in terms of increasing range.

%Input data
\subsection{Input data}
	We consider three data sets, consisting of road and public transit data. The road network is extracted from \osm \libref{osm}
	formatted data and transit data is given in the \gtfs \libref{gtfs} format.\\\\
	Our data sets represent the region around the german cities \freiburgR and \stuttgartR. Their road network is
	of similar size, while our transit data for \freiburgR only includes tram data, whereas the data for \stuttgartR
	also includes train and bus connections. The size of our transit network for \stuttgartR is about ten times the size of
	the network for \freiburgR.
	
	Furthermore, we include a road and transit network for the country \switzerlandR. The transit data consists of train, tram and
	bus connections. Both networks are about three times the size of the {\stuttgartR}s.

%OSM
\subsubsection{\osm}
	\osm data \libref{osm} is represented in a \xml structure describing
	\begin{itemize}
		\item[1.] \textit{nodes}, with an unique identifier and a coordinate given as pair of latitude and longitude;
		\item[2.] \textit{ways}, also with an unique identifier, consisting of multiple nodes referenced by their identifier;
		\item[3.] \textit{relations}, consisting of nodes, ways and other relations, representing relationships between the referenced data;
		\item[4.] \textit{tags} as key-value pairs, storing metadata about the other items.
	\end{itemize}
	\begin{lstlisting}[caption={\osm example data set.},label={osmExample},style={XMLStyle},mathescape={true},
		float,floatplacement=ht]
<?xml version='1.0' encoding='UTF-8'?>
<osm version="0.6">
  <bounds minlon="7.253190" minlat="47.299090" maxlon="9.246965" maxlat="48.751520"/>
  <node id="29764598" lat="47.8512831" lon="7.9230269"/>
  <node id="669209525" lat="47.8513215" lon="7.9231227"/>
  <node id="3993821274" lat="47.8513342" lon="7.923183"/>
  <node id="832450227" lat="47.8157938" lon="8.8487527">
    <tag k="highway" v="motorway_junction"/>
    <tag k="name" v="Kreuz Hegau"/>
  </node>
  <node id="100036455" lat="47.5728421" lon="8.0365409">
    <tag k="name" v="Niederhof"/>
    <tag k="traffic_sign" v="city_limit"/>
  </node>
  <way id="29764598">
    <nd ref="669209525"/>
    <nd ref="3993821274"/>
    <tag k="highway" v="motorway"/>
    <tag k="oneway" v="yes"/>
  </way>
  <relation id="56688">
    <member type="node" ref="29764598" role=""/>
    <member type="node" ref="669209525" role=""/>
    <member type="way" ref="29764598" role=""/>
    <tag k="name" v="Bus line 1"/>
    <tag k="network" v="VVW"/>
    <tag k="ref" v="1"/>
    <tag k="route" v="bus"/>
    <tag k="type" v="route"/>
  </relation>
</osm>
	\end{lstlisting}\quad\\
	A small \osm example data set is shown by \progref{osmExample}. Ways are used to represent roads consisting
	of nodes. Tags are used to describe metadata like speed limits for a road or whether it is a one-way street or not.
	However, the format also contains a lot of data not directly relevant for route planning, like shapes of buildings
	and outlines of public parks. Therefore, we filter \osm data and only keep relevant information.
	\begin{lstlisting}[caption={Tag filter for \osm ways.},label={osmFilter},style={FilterStyle},mathescape={true},
		float,floatplacement=ht]
--KEEP

#highways
highway=motorway
highway=trunk
highway=primary
highway=secondary
highway=tertiary
highway=residential
highway=living_street
highway=unclassified
highway=cycleway

#highwaylinks
highway=motorway_link
highway=trunk_link
highway=primary_link
highway=secondary_link
highway=tertiary_link
highway=residential_link

#non-standard
way=primary
way=seconday

--DROP

area=yes
train=yes
access=no
type=multipolygon
railway=platform
railway=station
highway=proposed
highway=construction
building=yes
building=train_station
	\end{lstlisting}\quad\\\\
	As we are only interested in the road network itself, we start by reading the ways. We filter them based on the tags
	described by \progref{osmFilter}. Ways having at least one of the key-value pairs described under \textit{$--$KEEP}
	and none of the pairs under \textit{$--$DROP} are kept, as they represent roads of the network. All other ways are
	rejected, as well as all relations.
	After that, we read the nodes and only keep nodes that occurred at least once in any of the ways that passed the filter.
	Our road network is then build using the remaining nodes as graph nodes, translating the ways into edges between the nodes.
	
	Ways with a positive \textit{oneway} tag are translated into edges only going into the given direction, else we generate
	both edges for both directions. The cost of an edge is computed as the time it needs to travel the direct distance between
	the source and destination coordinates (see \defref{asTheCrowFlies}) with a certain speed. The speed is determined either
	by a given \textit{maxspeed} tag or the average speed for the road type defined by the \textit{highway} tag.
	Therefore, we use the average speed references shown by \tableref{highwaySpeeds}.
	\begin{table}[ht]
	 	\begin{center}
	 		\phantom{v}\quad\\
	 		\begin{tabular}{|l|r|}
	 			\hline
	 			\multicolumn{1}{|c|}{tag value}	&\multicolumn{1}{c|}{$\o$ km/h}\\\hline

				motorway		&$120$\\
				trunk			&$110$\\
				primary		&$100$\\
				secondary		&$80$\\
				tertiary		&$70$\\
				motorway\_link	&$50$\\
				trunk\_link		&$50$\\
				primary\_link		&$50$\\
				secondary\_link	&$50$\\
				residential		&$50$\\
				unclassified		&$40$\\
				unsurfaced		&$30$\\
				road			&$20$\\
				cycleway		&$14$\\
				living\_street		&$7$\\
				service		&$7$\\\hline
			\end{tabular}
		\end{center}
		\caption{Average speed in km/h for a \osm way with the corresponding value for the \textit{highway} tag.}
		\label{highwaySpeeds}
	\end{table}\quad\\\\
	The size of the resulting road graphs (see \sectionref{roadGraphSec}) for all three data sets is reported in \tableref{osmSize}.
	As seen, filtering the \osm data sets beforehand reduces the size of data that is to be processed by $95\%$ to $97\%$.
	The road graphs have approximately two edges per node. This is due to most streets being two-way streets, thus generating
	two edges per connection between two nodes. Obviously, road junctions are, compared to the amount of nodes, rare and thus,
	multiple edges do only rarely share the same node. The in- and out-degree of nodes is extremely low, mostly either $1$ or $2$.
	\begin{table}[ht]
	 	\begin{center}
	 		\phantom{v}\quad\\
			\begin{tabular}{|l||r|r|r|r|}
				\hline
							&\multicolumn{2}{c|}{\osm data (MB)}	&\multicolumn{2}{c|}{Road graph}\\
							&\multicolumn{1}{c|}{raw}	&\multicolumn{1}{c|}{filtered}	&\multicolumn{1}{c|}{nodes}
								&\multicolumn{1}{c|}{edges}\\\hline
				\freiburgR		&$2\,260$	&$86$		&$743\,003$		&$1\,494\,883$\\
				\stuttgartR		&$2\,420$	&$118$	&$973\,142$		&$1\,950\,978$\\
				\switzerlandR	&$5\,530$	&$279$	&$2\,627\,645$	&$5\,226\,060$\\\hline
			\end{tabular}
		\end{center}
		\caption{Size of the \osm data sets, in megabyte (MB) before and after filtering, and the size of the resulting road graphs
			in amount of nodes $|V|$ and edges $|E|$.}
		\label{osmSize}
	\end{table}

%GTFS
\subsubsection{\gtfs}
	Blabla

%Experiments
\subsection{Experiments}
	Blabla

%Nearest neighbor computation
\subsubsection{Nearest neighbor computation}
	Blabla

%Uni-modal routing
\subsubsection{Uni-modal routing}
	Blabla

%Multi-modal routing
\subsubsection{Multi-modal routing}
	Blabla

%Summary
\subsection{Summary}
	Blabla